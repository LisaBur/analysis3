\chapter{Konvergenzsätze und $L^n$-Räume}
  \begin{example}
    Punktweise Konvergenz reicht nicht für Konvergenz der Integrale.\\
    Für $\epsilon > 0$ sei $f_{\epsilon}: \mathbb{R} \to \mathbb{R}, f_{\epsilon} = \dfrac{1}{2\epsilon} \chi_{[-\epsilon, \epsilon]}$\\
    Es gilt $f_{\epsilon}(x) = 0$ für $\epsilon < |x|$\\
    $\implies f(x) := \lim\limits_{\epsilon \downarrow 0} f_{\epsilon}(x) = \begin{cases}
      0 & \text{, für } x \neq 0\\
      \infty & \text{, für } x = 0
    \end{cases}$\\
    Weiter $\int f_{\epsilon} d\lambda^1 = \dfrac{1}{2\epsilon} \lambda^1([-\epsilon, \epsilon]) = 1 \ \forall \epsilon > 0$\\
    $\implies \int f d\lambda^1 = 0 < 1 = \lim\limits_{\epsilon \downarrow 0} f_{\epsilon} d\lambda^1$
  \end{example}

  \begin{theorem}[Lemma von Fatou]
    $f_k: X \to [0,\infty]$ Folge von $\mu$-messbaren Funktionen.\\
    Für $f: X \to \bar{\mathbb{R}}, f(x) = \liminf\limits_{k \to \infty} f_k(x)$ gilt:
    \begin{align*}
      \int f d\mu \leq \liminf\limits_{k \to \infty} \int f_k d\mu
    \end{align*}
  \end{theorem}

  \begin{proof}
    Definiere $g_k := \inf\limits_{j\geq k}f_j \implies g_{k+1} \geq g_k$ $\forall k\in\mathbb{N}$ und $\lim\limits_{k\to\infty}g_k = f$ \\
    $\overset{\text{Satz IV.10}}{\implies} \int f d\mu = \lim\limits_{k\to\infty} \int g_k d\mu \leq \liminf\limits_{k\to\infty} \int f_k d\mu$, da $g_k \leq f_k$ $\forall k\in\mathbb{N}$
  \end{proof}

  \begin{theorem}[Dominierte Konvergenz bzw. Satz von Lebesgue]
    $f_1, f_2, ...$ Folge von $\mu$-messbare Funktionen und $f(x) = \lim\limits_{k \to \infty} f_k(x)$ für $\mu$-fast alle $x \in X$. Es gebe eine integrierbare Funktion $g: X \to [0, \infty]$ mit $\sup\limits_{k \in \mathbb{N}} |f_k(x)| \leq g(x)$ für $\mu$-fast alle $x$. Fann ist $f$ integrierbar und $\int f d\mu = \lim\limits_{k \to \infty} \int f_k d\mu$.\\
    Es gilt sogar $||f_k \cdot f||_{L^1(y)} := \int |f_k -f| d\mu \to 0$
  \end{theorem}

  \begin{proof}
    Folge $2g-|f-f_k| \geq 0$ konvergiert punktweise fast überall gegen $2g$. \\
    $\overset{\text{Satz V.i}}{\implies} \limsup |\int f d\mu - \int f_k d\mu | \leq \limsup \int |f-f_k| d\mu \\ = \int 2g d\mu - \liminf \int (2g-|f-f_k|)d\mu \\ \leq 2g d\mu - \int \liminf (2g-|f-f_k|)d\mu = 0$
  \end{proof}

  \begin{remark}[Anwendung]
    Vergleich Riemann-$\int$ mit Lebesgue-$\int$\\
    Sei $I=[a,b]$ kompaktes Intervall, $f:I \to \mathbb{R}$ beschränkt. Unterteilungspunkte $a = x_0 \leq ... \leq x_N = b$ $\to$ Zerlegung $Z$ von $I$ mit Teilintervallen $I_j = [x_{j-1}, x_j]$\\
    $\bar{S}_Z(f) = \sum\limits_{j=1}^N (\sup\limits_{I_j} f) (x_j - x_{j-1}), \ \ \ \underbar{S}_Z(f)= \sum\limits_{j=1}^N (\inf\limits_{I_j} f)(x_j-x_{j-1})$\\
    Für Zerlegungen $Z_1, Z_2$ mit Verfeinerung $Z_1 \cup Z_2$\\
    $\implies \underbar{S}_{Z_1}(f) \leq \underbar{S}_{Z_1 \cup Z_2}(f) \leq \bar{S}_{Z_1 \cup Z_2}(f) \leq \bar{S}_{Z_2}(f)$\\
    $f$ heißt \textbf{Riemann-integrierbar} mit Integral $\int\limits_a^b f(x) dx = S$, falls gilt:\\
    $\sup\limits_Z \underbar{S}_Z(f) = \inf\limits_Z \bar{S}_Z(f) = S$
  \end{remark}

  \begin{theorem}
    $f: I \to \mathbb{R}$ beschränkt auf kompaktem Intervall $I=[a,b]$. Dann gilt:\\
    $f$ Riemann-integrierbar $\Leftrightarrow \lambda^1(\{x \in I \ | \ f \text{ ist nicht stetig in } x\}) = 0$\\
    In diesem Fall ist $f$ auch Lebesgue-integrierbar und die Integrale stimmen überein.
  \end{theorem}

  \begin{proof}
    Für Zerlegung $Z$ mit Teilintervallen $I_j = [x_{j-1},x_j]$, $1\leq j \leq N$.\\ Definiere Riemann-Treppenfunktion: \\ $\bar{f}_z(x) = \max\limits_{x\in I_j}\sup_{I_j} f \geq \limsup\limits_{y\to x} f(y)$, $\underline{f}_z(x) = \min\limits_{x\in I_j}\inf\limits_{I_j} f \leq \liminf\limits_{y\to x} f(y)$.  \\
    Sei $N_f(s) = \{ x\in I: \limsup\limits_{y\to x}f(y) - \liminf\limits_{y\to x}f(y) \geq s \}$ für $s > 0$. \\
    Sind $Z_1$, $Z_2$ bel. Zerlegungen $\implies \bar{f}_{Z_2}(x) - \underline{f}_{Z_1}(x) \geq s$ $\forall x\in N_f(s)$. \\
    $\implies \bar{S}_{Z_2}(f) - \underline{S}_{Z_1}(f) = \int\limits_{I} (\bar{f}_{Z_2}(x)-\underline{f}_{Z_1}(x)) d\lambda^1 \geq s \lambda^1 (N_f(s))$ \\
    Ist $f$ Riemann-integrierbar, so bilde Infimum über alle $Z_1$, $Z_2$ und schließe $\lambda^1 (N_f(s))=0$ $\forall s>0$ $\implies$ "$\implies$". \\
    Sei nun $f$ $\lambda^1$-fast überall stetig und $Z_i$ Folge von Zerlegungen mit Feinheit \\ $\delta_i := \max\limits_{1\leq j \leq N_i} |x_{i,j}-x_{i,j-1}| \to 0$ \\
    Ist $f$ stetig in $x$, so folgt $\bar{f}_{Z_i} (x) \geq \inf\limits_{|y-x| \leq \delta_i } f(y) \to f(x)$ und $\underline{f}_{Z_i} (x) \geq \inf\limits_{|y-x| \leq \delta_i } f(y) \to f(x)$ mit $i\to\infty$ $\implies \bar{f}_{Z_i}$, $\underline{f}_{Z_i}$ konvergiert punktweise $\lambda^1$-fast überall gegen $f$ $\implies f$ ist $\lambda^1$-messbar nach Kapitel I. \\
    Aus $|\bar{f}_{Z_i}|$, $|\underline{f}_{Z_i}| \leq \sup\limits_{I} |f| < \infty$ folgt aus Satz V.2 $\bar{S}_{Z_i}(f) = \int\limits_{I} \bar{f}_{Z_i} d\lambda^1 \to \int\limits_{I} f d\lambda^1$, \\
    $\underline{S}_{Z_i}(f) = \int\limits_{I} \underline{f}_{Z_i} d\lambda^1 \to \int\limits_{I} f d\lambda^1 \implies f$ ist Riemannint. mit $\int\limits_{a}^b f(x) dx = \int\limits_{I} f d\lambda^1$.
  \end{proof}

  \begin{theorem}
    $X$ metrischer Raum, $\mu$ Maß auf $Y$ und $f:X \times Y \to \mathbb{R}$ mit $f(x, \cdot)$ integrierbar bzgl. $\mu \ \forall x \in X$.\\
    Betrachte $F: X \to \mathbb{R}, F(x) = \int f(x,y) d\mu(y)$\\
    Sei $f(\cdot, y)$ stetig in $x_0 \in X$ für $\mu$-fast alle $y \in Y$. Weiter gebe es eine $\mu$-integrierbare Funktion $g: Y \to [0, \infty]$, so dass für alle $x \in X$ gilt: $|f(x,y)| \leq g(y) \ \forall y \in Y \setminus N_X$ mit einer $\mu$-Nullmenge $N_x$.\\
    Dann ist $F$ stetig in $x_0$.
  \end{theorem}

  \begin{proof}
    Sei $\alpha_k \to x_0$ Folge $\implies \exists \mu$-Nullmenge $N$, so dass $\forall y\in y\setminus N$ gilt: \\
    $f(x_k,y) \to f(x_0,y)$ und $|f(x_k,y)| \leq g(y)$ $\forall k\in\mathbb{N}$. \\
    $\overset{\text{Satz V.2}}{\implies} F(x_0) = \int f(x_0,y) d\mu(y) = \lim\limits_{k\to\infty} \int f(x_k,y) d\mu(y) = \lim\limits_{k\to\infty} F(x_k)$.
  \end{proof}

   \sidenote{Vorlesung 14}{18.12.20}

  \begin{theorem}
    Sei $I \subseteq \mathbb{R}$ offenes Intervall, $\mu$ Maß auf $Y$ und $f: I \times Y \to \mathbb{R}$ mit $f(x, \cdot)$ integrierbar bzgl. $\mu$ für alle $x \in I$.\\
    Setze $F: U \to \mathbb{R}, F(x) = \int f(x,y) d\mu(y)$\\
    Es sei $f(\cdot, y)$ in $x_0$ differenzierbar für $\mu$-fast alle $y \in Y$ und es existiere $g: Y \to [0, \infty]$ $\mu$-integrierbar mit
    \begin{align*}
      \dfrac{|f(x,y) - f(x_0, y)|}{|x-x_0|} \leq g(y) \ \forall x\in I \ \forall y \in Y \setminus N_x
    \end{align*} 
    mit einer $\mu$-Nullmenge $N_x$. Dann folgt:
    \begin{align*}
      F'(x_0) = \int \dfrac{\partial f}{\partial x} (x_0, y) d\mu(y)
    \end{align*}
  \end{theorem}
  \begin{proof}
    Zu jeder Folge $x_k \to x_0$ existiert $\mu$-Nullmenge $N\subset Y$, sodass $\forall y\in Y\setminus N$ gilt: \\
    $$\lim\limits_{k\to\infty} \frac{f(x_k,y)-f(x_0,y)}{x_k-x_0} = \frac{\partial f}{\partial_x}(x_0,y)$$ \\
    $$\frac{|f(x_k,y)-f(x_0,y)|}{|x_k-x_0|} \leq g(y)$$ 
    $$\overset{\text{Satz V.2}}{\implies} \frac{F(x_k)-F(x_0)}{x_k-x_0} = \int \frac{f(x_k,y)-f(x_0,y)}{x_k-x_0} d\mu(y) \to \int \frac{\partial f}{\partial x}(x_o,y) d\mu(y)$$
  \end{proof}

  \newpage

  \begin{lemma}
    $\script{U} \subseteq \mathbb{R}^n$ offen, $\mu$ Maß auf $Y$ und $f: \script{U} \times Y \to \mathbb{R}$ mit $f$ integrierbar bzgl. $\mu \ \forall x \in \script{U}$. Betrachte $F: \script{U} \to \mathbb{R}, F(x) = \int f(x,y) d\mu(y)$\\
    Es gebe eine $\mu$-Nullmenge $N \subseteq Y$, so dass $\forall y \in Y \setminus N$ gilt:
    \begin{align*}
      f(\cdot, y) \in C^1(\script{U}) \text{ und } |D_x f(x,y)| \leq g(y) \text{ mit } g: Y \to [0, \infty] \text{ integrierbar}
    \end{align*}
    $\implies F \in C^1(\script{U})$ und $\forall x \in \script{U}$ gilt:
    \begin{align*}
      \dfrac{\partial F}{\partial x_i}(x) = \int \dfrac{\partial f}{\partial x_i}(x,y) d\mu(y)
    \end{align*}
  \end{lemma}

  \begin{proof}
    Nach Vor. gilt $\forall y\in Y$ mit Ausnahme einer $\mu$-Nullmenge $N$: \\
    $$ \frac{|f(x+hc_i, y)-f(x,y)|}{h} \leq \int\limits_{0}^1 \left| \frac{\partial f}{\partial x_i} (x+tc_i,y) \right|  dt \leq g(y) $$
    Satz V.5 $\implies$ $F$ ist in allen $x\in \script{U}$ nach $x_i$ partiell differenzierbar mit gewünschten Ableitung. \\
    Satz V.4 $\implies$ partielle Ableitungen sind stetig auf U $\implies F\in C^1(\script{U})$. 
  \end{proof}

  \begin{example}
    \begin{align*}
      \int\limits_0^{\infty} \dfrac{\sin(x)}{x} dx = ? \ \ \ \ \text{Betrachte $F: [0, \infty] \to \mathbb{R}, F(t) = \int\limits_0^{\infty} e^{-tx} \dfrac{\sin{x}}{x} dx$}
    \end{align*}
    $f(t,x) := e^{-tx} \dfrac{\sin(x)}{x}$ hat für $t \geq \delta$ die Abschätzungen\\
    $|f(t,x)|, |\partial_t f(t,x)| \leq e^{-\delta x} =: g(x) \in L^1([0, \infty))$\\
    Lemma V.6 $\implies \forall t > 0$ gilt: 
    \begin{align*}
      F'(t) &= \int\limits_0^{\infty} e^{-tx} (-\sin{x}) dx\\
            &= [e^{-tx} \cos{x}]_{x=0}^{x=\infty} + t \int\limits_0^{\infty} e^{-tx} \cos{x} dx\\
            &= -1 + t^2 \int\limits_0^{\infty} e^{-tx} \sin{x} dx\\
            &= -1 - t^2 F'(t)
    \end{align*}
    $\implies F'(t) = \dfrac{-1}{1+t^2}$\\
    Weiter ist $\lim\limits_{t\to\infty} f(t,x) = 0$ $\forall x>0$ mit Majorante $\text{e}^{-x}$  \\
    Satz V.2 $\implies \lim\limits_{t\to\infty} F(t) = 0 \implies F(t) = \frac{\pi}{2}-\arctan{t}$ $\forall t>0$. \\
    Für $t>0$ und $0<r<R<\infty$ gilt mit $\sin{x} = \text{Im}(\text{e}^{ix})$: \\
    $ \int\limits_r^R \text{e}^{-tx} \frac{\sin{x}}{x}dx = \text{Im}$ $\int\limits_r^R \text{e}^{(i-t)x}\frac{dx}{x} = \text{Im}$ $\frac{\text{e}^{(i-t)x}}{(i-t)x} |_{x=r}^{x=R} + \text{Im}$ $\int\limits_r^R \frac{\text{e}^{(i-t)x}}{(i-t)}\frac{dx}{x^2}$ \\
    Mit $R\to\infty$ sehen wir im Fall $t=0$ die Existenz von $F(0) = \lim\limits_{k\to\infty}\int\limits_0^R \frac{\sin{x}}{x}dx$. Weiter folgt für $t\geq 0$ die Abschätzung $\left| \int\limits_r^\infty \text{e}^{-tx} \frac{\sin{x}}{x}dx \right| \leq \frac{2}{r}$ (denn $|i-t|\geq 1$). Somit folgt $\left| F(0) - F(t) \right| \leq \left| \int\limits_0^r (1-\text{e}^{-tx}) \frac{\sin{x}}{x} dx \right| + \frac{4}{r}$. \\
    Satz V.2 $\implies \limsup\limits_{t\to 0} \left| F(0)-F(t)\right| \leq \frac{4}{r}$. \\
    Mit $r\to\infty$ folgt 
    $$\int\limits_0^{\infty} \dfrac{\sin(x)}{x} dx = F(0) = \lim\limits_{t\to 0}F(t) = \lim\limits_{t\to 0} \left( \dfrac{\pi}{2}-\arctan{t} \right) = \dfrac{\pi}{2}$$
  \end{example}

  \begin{definition}[$L^p$-Norm]
    Für $\mu$-messbares $f: X \to \bar{\mathbb{R}}$ und $1 \leq p \leq \infty$ setzen wir
    \begin{align*}
      ||f||_{L^p(\mu)} := \begin{cases}
        (\int |f|^p d\mu)^{1/p} & \text{, für } 1\leq p < \infty\\
        \inf\{s>0 \ | \ \mu(\{|f| > s\})=0\} & \text{, für } p = \infty
      \end{cases}
    \end{align*}
    auf $\script{L}^p(\mu) = \{f:X \to \bar{\mathbb{R}} \ | \ f \mu-\text{messbar }, ||f||_{L^p(\mu)} < \infty\}$\\
    Betrachte Äquivalenzrelation $f\sim g \Leftrightarrow f(x) = g(x)$ für $\mu$-fast alle $x \in X$, und definiere den $\bm{L^p}$\textbf{-Raum} durch $\script{L}^p(\mu)/_{\sim}$.
  \end{definition}

  \begin{definition}
    Für $E \subseteq X$ messbar und $f: E \to \bar{\mathbb{R}}$ sei $f_0: X\to \bar{\mathbb{R}}$ die \textbf{Fortsetzung} mit $f_0(x)=0 \ \forall x \in X \setminus E$. Wir setzen dann
    \begin{align*}
      \script{L}^p(E) := \{f:E \to \bar{\mathbb{R}} \ | \ f_0 \in \script{L}^p(\mu)\}
    \end{align*}
    und $L^p(E,\mu) := \script{L}^p(E)/_{\sim}$.
  \end{definition}

  \begin{proposition}
    Für $1 \leq p \leq \infty$ ist $(L^p(\mu), ||\cdot||_{L^p(\mu)})$ ein normierter Vektorraum. Insbesondere gelten für $\lambda \in \mathbb{R}$ und $f,g \in L^p(\mu)$:
    \begin{enumerate}
      \item $||f||_{L^p} = 0 \implies f = 0$ $\mu$-fast überall
      \item $f \in L^p(\mu), \lambda \in \mathbb{R} \implies \lambda f \in L^p(\mu), \ ||\lambda f||_{L^p} = |\lambda| \ ||f||_{L^p}$
      \item $f,g \in L^p(\mu) \implies f+g \in L^p(\mu)$ und $||f+g||_{L^p} \leq ||f||_{L^p} + ||g||_{L^p}$
    \end{enumerate}
  \end{proposition}

  \begin{proof}
    Sei $1 \leq p >\infty \implies ||f||_{L^p}$ ist wohldefiniert nach Satz IV.6. \\
    1. folgt aus Lemma IV.8 \\
    2. folgt aus Linearität des Integralt. \\
    $t \mapsto t^p$ ist konvex auf $[0,\infty) \implies |f+g|^p = 2^p \left| \frac{f+g}{2}\right| ^p \leq 2^{p-1} (|f|^p + |g|^p) \implies$ Aus $f,g \in L^p(\mu)$ folgt $f+g\in L^p(\mu)$. \\
    $\Delta$-Ungleichung folgt später. \\
    \item[$p=\infty$:] 
    \item[1)] ist klar
    \item[2)] O.E. $\lambda >0 \rightarrow \{|\lambda f| > \lambda s\} = \{|f| > s \}$
    \item[3)] $\{|f+g| > s_1+s_2 \}\subset \{|f|>s_1 \}\cup \{|g|>s_2 \}$
  \end{proof}

  \begin{lemma}[Youngsche Ungleichung]
    Für $1 < p,q < \infty$ mit $\dfrac{1}{p} + \dfrac{1}{q} = 1$ und $x,y \geq 0$ gilt: \ \ $xy \leq \dfrac{x^p}{p} + \dfrac{y^q}{q}$
  \end{lemma}

  \begin{proof}
    Sei $y\geq 0$ fest und $f(x) = \frac{1}{p}x^p +\frac{1}{q}y^q -xy \implies f'(x) = \alpha^{p-1}-y \begin{cases} <0 \text{ für } x<y^{\frac{1}{p-1}} \\ > 0 \text{ für } x>y^{\frac{1}{p-1}}\end{cases} $ \\
    $\implies \forall x\geq 0$: $f(x) \geq f(y^{\frac{1}{p-1}}) = \frac{1}{p} y^{\frac{p}{p-1}}+\frac{1}{q} y^{\frac{p}{p-1}} - y^{\frac{p}{p-1}} = 0$
  \end{proof}

  \begin{theorem}[Höldersche Ungleichung]
    Für $\mu$-messbare $f,g: X \to \bar{\mathbb{R}}$ gilt: \ \ $|\int fg d\mu| \leq ||f||_{L^p} ||g||_{L^p}$,\\
    falls $1 \leq p,q \leq \infty$ mit $\dfrac{1}{p} + \dfrac{1}{q} = 1$
  \end{theorem}

  \begin{proof}
    O.E. $f,g \geq 0$ und $||f||_{L^p} = ||g||_{L^q} = 1$ (sonst $\tilde{f} = \dfrac{f}{||f||_{L^p}}$, $\tilde{g} = \dfrac{g}{||g||_{L^q}}$) \\
    Lemma V.10 $\implies \int fg d\mu \leq \int \left(\frac{f^p}{p}+\frac{g^q}{q} \right)d\mu = 1 = ||f||_{L^p} ||g||_{L^q} \\ \implies$ Beh. für $1 < p,q < \infty$. \\
    Fall $p=1, q =\infty$ folgt sofort aus Satz IV.6.
  \end{proof}

  \begin{theorem}[Minkowski-Ungleichung]
    Für $f,g \in L^p(\mu)$ mit $1 \leq p \leq \infty$ gilt: \ \ $|| f+g ||_{L^p} \leq ||f||_{L^p} + ||g||_{L^p}$
  \end{theorem}

  \begin{proof}
    $$ ||f+g||^p_{L^p}$$ $$= \int |f+g|^p d\mu \leq |f| |f+g|^{p-1} d\mu + \int |g| |f+g|^{p-1}d\mu$$ $$\leq ||f||_{L^p}||f+g||_{L^p}^{p-1} + ||g||_{L^p}||f+g||_{L^p}^{p-1}$$ $$ \overset{\text{Kürzen}}{\implies} ||f+g||_{L^p} \leq ||f||_{L^p} + ||g||_{L^p}$$
  \end{proof}

  \begin{lemma}
    Sei $1 \leq p < \infty$ und $f_k = \sum\limits_{j=1}^k u_j$ mit $u_j \in L^p(\mu)$. Falls $\sum\limits_{j=1}^k ||u_j||_{L^p} < \infty$, so gelten:
    \begin{enumerate}[label=\roman*)]
      \item $\exists \ \mu$-Nullmenge $N$: $f(x) = \lim\limits_{k \to \infty} f_k(x) \ \forall x \in X \setminus N$ ex.
      \item mit $f := 0$ auf gilt $f \in L^p(\mu)$
      \item $||f - f_k||_{L^p} \to 0$ mit $k \to \infty$
    \end{enumerate}
  \end{lemma}

  \begin{proof}
    Betrachte $g_k := \sum\limits_{j=1}^k |u_j|$, $g:= \sum\limits_{j=1}^\infty |u_j|$ \\
    Es gilt: $g_1 \leq g_2 \leq ...$ und $g_k(x) \to g(x) \in [0,\infty]$ mit $k\to\infty$ $\forall x\in X$ \\
    Satz IV.10 $\implies ||g||_{L^p} = \lim\limits_{k\to\infty} ||g_k||_{L^p} \overset{\text{Satz V.12}}{\leq} \sum\limits_{j=1}^\infty ||u_j||_{L^p} < \infty$ \\
    Lemma IV.8 $\implies N:=\{g=\infty \}$ ist $\mu$-Nullmenge. \\ Für $x\in X\setminus N$ ist $\sum\limits_{j=1}^\infty u_j(x)$ absolut konvergent.$\implies (f_k(x))$ ist Cauchy-Folge in $\mathbb{R} \\
    \implies f(x) = \lim\limits_{k\to\infty} f_k(x)$ existiert $\forall x\in X\setminus N$. \\ Weiter ist $|f_k|^p \leq |g|^p \in L^1(\mu)$, sowie $|f-f_k|^p \leq 2^{p-1}(|f|^p+|f_k|^p) \leq 2^p g^p$ \\
    Satz V.2 $\implies f\in L^p(\mu)$ und $||f-f_k||_{L^p} \to 0$
  \end{proof}

  \sidenote{Vorlesung 15}{21.12.20}

  \begin{theorem}[Satz von Riesz-Fischer]
    $(L^p(\mu), ||\cdot||_{L^p})$ ist vollständig, also ein Banachraum. $(1 \leq p \leq \infty)$
  \end{theorem}

  \begin{proof}
    Sei $f_k \in L^p(\mu)$ Cauchyfolge bzgl $||\cdot ||_{L^p}$. \\
    Es reicht zu zeigen: $\exists$ Teilfolge, welche in $L^p(\mu)$ konvergiert.
    \item[1)] $1 \leq p < \infty$: \\
    Nach Wahl einer Teilfolge sei $||f_{k+1}-f_k||_{L^p} \leq 2^{-k}$ $\forall k\in\mathbb{N}$. \\
    Mit $f_0 := 0 \implies f_k = \sum\limits_{j=1}^k u_k$ mit $u_j = f_j - f_{j-1}$. \\
    Lemma V.13 $\implies$ $f_k$ konvergiert in $L^p(\mu)$ bzw. punktweise $\mu$-fast überall gegen $f\in L^p(\mu) \implies$ Beh.
    \item[2)] $p = \infty$ \\
    Wegen $\left| ||f_k||_{L^\infty} - ||f_l||_{L^\infty} \right| \leq ||f_k - f_k||_{L^\infty}$ existiert $\lim\limits_{k\to\infty} ||f_k||_{L^\infty}$. \\
    Die Mengen $N_k = \{|f_k| > ||f_k||_{L^\infty}  \}$ sowie $N_{k,l} = \{|f_k-f_l| > ||f_k-f_l||_{L^\infty} \}$ haben $\mu$-Maß Null $\implies N = \bigcup\limits_{k=1}^\infty N_k \cup \bigcup\limits_{k,l=1}^\infty N_{k,l}$ ist $\mu$-Nullmenge. \\
    Für $x\in X\setminus N$ gilt: $|f_k(x)-f_l(x)|\leq ||f_k-f_l||_{L^\infty} < \epsilon$ für $k,l \geq k(\epsilon) \\ 
    \implies f(x) = \lim\limits_{k\to\infty} f_k(x)$ ist definiert $\forall x\in X\setminus N$. \\
    Weiter gilt für $x\in X\setminus N$:\\
    $|f(x)| = \lim\limits_{k\to\infty}|f_k(x)| \leq \lim\limits_{k\to\infty} ||f_k||_{L^\infty}$ und $|f_k(x) - f(x)| = \lim\limits_{l\to\infty} |f_k(x)-f_l(x)| \leq \lim\limits_{l\to\infty} ||f_k-f_l||_{L^\infty} \leq \epsilon$ für $k\geq k(\epsilon)$ \\
    $\implies ||f||_{L^\infty} \leq \lim\limits_{k\to\infty} ||f_k||_{L^\infty} < \epsilon$ und $||f_k-f||_{L^\infty} \to 0$ mit $k\to\infty$.
  \end{proof}

  \begin{lemma}
    Konvergiert $f_k$ gegen $f$ in $L^p(\mu)$, so konvergiert eine Teilfolge $f_{k_j}$ punktweise $\mu$-fast überall gegen $f$.
  \end{lemma}  
  
  \begin{example}
    Im Fall $p < \infty$ kann im Allgemeinen nicht auf die Wahl einer Teilfolge verzichtet werden:\\
    Jedes $n \in \mathbb{N}$ besitzt die eindeutige Darstellung $n=2^k+j$ mit $k \in \mathbb{N}_0, 0 \leq j < 2^k$\\
    Definiere damit $f_n:[0,1] \to \mathbb{R}, f_n(x) = \begin{cases}
      1 & \text{, falls } j \cdot 2^{-k} \leq x \leq (j+1) 2^{-k}\\
      0 & \text{, sonst}
    \end{cases}$\\
    $\int\limits_0^1 f_n(x) dx = 2^{-k} < \dfrac{2}{n} \to 0$ mit $n \to \infty$\\
    Andererseits: $\limsup\limits_{n \to \infty} f_n(x) = 1 \ \forall x \in [0,1)$, denn zu $x \in [0,1), k \in \mathbb{N}$ können wir\\
    $j \in \{0,1,...,2^k-1\}$ wählen mit $j \cdot 2^{-k} \leq x < (j+1) 2^{-k}$\\
    $\implies f_n(x) = 1$ für $n = 2^k+j$\\
    $\implies$ Folge konvergiert nicht punktweise $\lambda^1$-fast überall gegen $0$.
  \end{example}

  \begin{remark}
    Jetzt betrachten wir $\mu=\lambda^n$ im $\mathbb{R}^n$.\\
    Im $\mathbb{R}^n$ haben wir eine Metrik.
  \end{remark}

  \begin{definition}
    Der \textbf{Träger} einer Funktion $f:\Omega \to \mathbb{R}, \Omega \subseteq \mathbb{R}^n$ offen, ist die Menge 
    \begin{align*}
      spt(f) = \overline{\{x \in \mathbb{R} \ | \ f(x) \neq 0\}}
    \end{align*}
    Der Raum der stetigen Funktionen mit kompaktem Träger in $\Omega$ wird mit $C_c^0(\Omega)$ bezeichnet.\\
    Für $K \subseteq \Omega$ kompakt sei $dist(\cdot, K): \mathbb{R}^n \to [0, \infty), dist(x, K) = \inf\limits_{z \in K} ||x - z||$ die \textbf{Abstandsfunktion} von K.\\
    Wir benötigen:
    \begin{enumerate}
      \item $dist(\cdot, K)$ ist Lipschitz-stetig mit Konstante $1$
      \item $dist(\mathbb{R}^n \setminus \Omega, K) = \inf\limits_{x \in \mathbb{R}^n \setminus \Omega} dist(x, K) > 0$
    \end{enumerate}
  \end{definition}

  \begin{theorem}
    Sei $\Omega \subseteq \mathbb{R}^n$ offen und $1 \leq p < \infty$. Dann existiert zu jedem $f \in C^p(\Omega)$ eine Folge $f_k \in C_c^0(\Omega)$ mit $|| f_k - f||_{L^p(\Omega)} \to 0$ mit $k \to \infty$.
  \end{theorem}

  \begin{proof}
    \item[1)] $f=\psi_E$ mit $E\subset \Omega$ messbar mit $\lambda^n(E) < \infty$\\
    Sei $\epsilon >0$. Lemma III.6 $\implies$ $\exists K\subset E$ kompakt mit $\lambda^n(E\setminus K) < \frac{\epsilon}{2}$ \\
    Setze $f_\rho: \Omega \to [0,1]$, $f_\rho (x) = \left( 1-\frac{dist(x,K)}{\rho}\right) ^+ \\ 
    \overset{\text{1),2)}}{\implies} f_\rho \in C^0(\Omega)$, $spt f_\rho = \{x: dist(x,K)\leq \rho \}$ ist kompakte Teilmenge von $\Omega$ für $\rho$ hinreichend klein. $\implies f_\rho \in C^0_c (\Omega)$ für $\rho << 1$ und $f_\rho = f$ auf $K$. \\
    $$\implies \int\limits_{\Omega} |f_\rho -f |^p d\lambda^n \leq 2^{p-1} \int\limits_{\Omega\setminus K} (|f_\rho)|^p + |f|^p d\lambda^n$$
    $$ \leq 2^{p-1} (\lambda^n (\{ 0< dist(\cdot, K)\leq \rho\}) + \lambda^n(E\setminus K))$$ 
    $$ < 2^{p-1}\epsilon \text{ für } \rho \text{ hinreichend klein }$$ 
    $\implies$ Beh. 
    \item[2)] $f$ beliebig, $f\in L^p(\Omega)$ \\
    O.E. $f \geq 0$ sonst betrachte $f^+$ und $f^-$. \\
    Nach Satz IV.9 existiert Folge von Treppenfunktionen $f_1 \leq f_2 \leq ...$ mit \\ $\lim\limits_{k\to\infty} f_k(x) = f(x)$ $\forall x\in \Omega$. \\
    Satz V.2 $\implies f_k \to f$ in $L^p(\Omega)$. \\
    Verwende dazu Majorante $|f-f_k|^p \leq 2^{p-1}(|f|^p+|f_k|^p) \leq 2^p |f|^p \in L^p(\Omega)$. \\
    Für die Treppenfunktion $f_k$ gilt nach Lemma IV.7
    $$ \lambda^n (\{f_k \geq s \}) \leq \frac{1}{s^p} \int\limits_\Omega |f_k|^p d\lambda^n \leq \frac{1}{s^p} \int\limits_\Omega |f|^p d\lambda^n < \infty$$ 
    Da $f_k$ endliche Linearkombination von charakteristischen Funktionen ist, folgt die Behauptung. 
  \end{proof}

  \begin{remark}
    $BC^0(\Omega)$ bezeichnet die Menge der beschränkten, stetigen Funktionen auf $\Omega$. Mit Supremumsnorm $||\cdot||_{\sup}$ ist diese ein Banachraum. Für $f\in  BC^0(\Omega)$ gilt \\
    $$\{ |f|>s \} \neq \varnothing \Leftrightarrow \lambda^n(\{ |f|>s \}) > 0$$
    $\implies ||f||_{L^\infty} = ||f||_{\sup}$ $\forall f\in BC^0(\Omega)$. \\
    Angenommen $f\in L^\infty(\Omega)$ kann durch $f_k \in BC^0(\Omega)$ bzgl der $L^\infty$-Norm approximiert werden, dann wäre $f_k$ eine Cauchyfolge bzgl $\sup$-Norm \\
    $\implies \exists \tilde{f} \in BC^0(\Omega)$ mit $f_k\to \tilde{f}=f$ fast überall, aber $L^\infty$-Funktion ohne stetigen Repräsentanten.
    \item[] \underline{Fourier-Reihen:} \\
    Betrachte Funktionen $f\in L^2(I,\mathbb{C})$ mit $I =(-\pi, \pi) \Leftrightarrow \text{Re}(f)$, $\text{Im}(f) \in L^2(I)$ \\
    Satz V.14 $\implies L^2(I,\mathbb{C})$ ist vollständig bzgl $L^2$-Norm. Also ein Hilbertraum mit dem hermitischen Skalarprodukt $$<f,g>_{L^2} = \int\limits_{-\pi}^\pi f(x) \overline{g(x)}dx$$
    $w_k (x) = \frac{1}{\sqrt{2\pi}} \text{e}^{ikx}$, $k\in\mathbb{Z}$ bilden ein Orthonormalsystem und spannen den Raum $\mathbb{P}$ der Trigonometrischen Polynome auf. \\
    Das $n$-te Fourierpolynom $f_n$ ist definiert als die Orthonormalprojektion von $f$ auf den Raum $\mathbb{P}^n$ der trigonometrischen Polynome von Grad $\leq n$. , Also\\
    $$ f_n = \sum\limits_{k=-n}^n <f,w_k>_{L^2} w_k = \sum\limits_{k=-n}^n \hat{f}(k) \text{e}^{ik}$$
    $\hat{f}(k) = \frac{1}{\sqrt{2\pi}} <f,w_k>_{L^2} = \frac{1}{2\pi} \int\limits_{-\pi}^\pi f(x) \text{e}^{-ikx} dx$. Die Folge $f_n$ heißt Fourier-Reihe von $f$. \\
    Wegen $f-f_n \perp_{L^2} \mathbb{P}_n$ gilt $\forall p\in \mathbb{P}_n$ : \\
    \begin{align*}
    (\ast) & & |f-p||_{L^2}^2 = ||(f-f_n)+(f_n-p) ||_{L^2}^2 = ||f-f_n||_{L^2}^2 + ||f_n-p||_{L^2}^2
    \end{align*}
	Mit $p = 0$ folgt die Besselsche Ungleichung 
	\begin{align*}
	(\ast\ast) & & 2\pi \sum\limits_{k=-n}^n |\hat{f}(k)|^2 = ||f_n||^2_{L^2} \overset{(\ast)}{\leq} ||f||_{L^2}^2 & &  \forall n\in\mathbb{N}_0
	\end{align*}
	$\implies f_n$ ist eine $L^2$-Cauchyfolge, denn für $m\geq n$ folgt \\  $||f_m-f_n||^2_{L^2} = 2\pi \sum\limits_{k=n+1}^m |\hat{f}(k) |^2 + 2\pi \sum\limits_{k=-m}^{-(n+1)} |\hat{f}(k)|^2 < \epsilon$ für $n$ groß genug. \\
	$\overset{\text{Riesz-Fischer (Satz V.14)}}{\implies} f_n$ konvergiert in $L^2(I,\mathbb{C})$ gegen  eine Funktion $\in L^2(I, \mathbb{C})$. \\
	Frage: Konvergiert $f_n$ gegen $f$?
	
  \end{remark}

  \begin{theorem}
    Für $f \in L^2(I, \mathbb{C})$ konvergiert $f_n$ gegen $f$ in $L^2(I, \mathbb{C})$? (bezieht sich auf Bem. vorher)
  \end{theorem}
  \begin{proof}
    Wegen $(\ast)$ gilt $$ ||f-f_n||_{L^2} = \min\limits_{p\in\mathbb{P}_n} ||f-p||_{L^2}$$
    \item[]\underline{z.z.} $\mathbb{P}$ liegt dicht in $L^2(I,\mathbb{C})$ \\
    Satz V.17 $\implies C_c^0(I,\mathbb{C})$ dicht in $L^2(I,\mathbb{C})$ und jede stetige Funktion kann gleichmäßig und damit auch in $L^2$ durch stückweise konstante Funktionen approximiert werden. Für $f$ stückweise $C^1$ haben wir in Analysis II gezeigt: $f_n \to f$ gleichmäßig, also auch in $L^2$. 
  \end{proof}

  \begin{remark}
    Sei $\ell^2(\mathbb{C})$ der Raum aller komplexen Folgen $c = (c_k)_{k \in \mathbb{Z}}$ mit $||c||_{\ell^2}^2 = 2 \pi \sum\limits_{k \in \mathbb{Z}} |c_k|^2 < \infty$\\
    $\ell^2(\mathbb{C})$ ist vollständig (folgt aus Riesz-Fischer angewandt auf das Zählmaß auf $\mathbb{Z}$)
  \end{remark}

  \begin{lemma}
    Die Abbildung $\script{F}: (L^2(I, \mathbb{C}), ||\cdot||_{L^2}) \to (\ell^2(\mathbb{C}), ||\cdot||_{\ell^2}), \script{F}(f) = (\hat{f}(k))_{k \in \mathbb{Z}}$ ist eine Isometrie von Hilberträumen.
  \end{lemma}

  \begin{proof}
    Aus Satz V.18 folgt mit $p=0$ in $(\ast)$: $$  ||\script{F}(f)||_{l^2}^2 = 2\pi \sum\limits_{k=-\infty}^\infty |\hat{f}(k)|^2 = \lim\limits_{n\to\infty} ||f_n||_{L^2}^2 \overset{(\ast)}{=} \lim\limits_{n\to\infty} \left( ||f||_{L^2}^2 - ||f-f_n||^2_{L^2}\right) = ||f||_{L^2}^2$$ 
    $\implies \script{F}$ ist isometrisch und damit injektiv. \\
    Für $c\in l^2(\mathbb{C})$ ist $f_n = \sum\limits_{k=-n}^n c_k \text{e}^{-ikx}$ eine Cauchyfolge in $L^2(I,\mathbb{C})$ und konvergiert nach Riesz-Fischer gegen $f\in L^2(I,\mathbb{C}) \implies \script{F}(f) = c \implies \script{F}$ ist surjektiv. 
  \end{proof}

  \begin{remark}
    Die Konvergenz der Fourierreihe ist ein Spezialfall des Spektralsatzes für selbstadjungierte Operatoren. Dieser verallgemeinert die Diagonalisierbarkeit symmetrischer Matrizen (siehe LA) auf $\infty$-dimensionalen Räume.\\
    Hier ist der Operator $H = -\dfrac{d^2}{dx^2}$ ein Endomorphismus auf $C_{Per}^{\infty}(I)$\\
    $H:C_{Per}^{\infty}(I) \to C_{Per}^{\infty}(I), Hf=-\dfrac{d^2f}{dx^2}$\\
    Part. Int. $\implies \langle Hf,g \rangle_{L^2} = \langle f,Hg \rangle_{L^2} \ \forall f,g \in C_{Per}^{\infty}(I)$ sowie $\langle Hf, f \rangle_{L^2} = ||\dfrac{df}{dx}||_{L^2}^2 \geq 0$\\
    Die $w_k$ sind Eigenfunktionen von den Eigenvektoren $\lambda_k = k^2$:\\
    $Hw_k = \lambda^2 w_k \ \forall k \in \mathbb{Z}$\\
    Satz V.18: Der von den Eigenfunktionen $w_k$ aufgespannte Raum ist dicht in $L^2(I, \mathbb{C})$
  \end{remark}

  \sidenote{Vorlesung 16}{08.01.21}
  \begin{theorem}[Vitali]
    Sei $f_n \in L^p(\mu), 1 \leq p \leq \infty$, eine Folge mit $f_n \to f$ punktweise fast überall. Dann sind folgende Aussagen äquivalent:
    \begin{enumerate}[label=\alph*)]
      \item $f \in L^p(\mu)$ und $||f_n - f||_{L^p} \to 0$
      \item Mit $\lambda(A) = \limsup \int\limits_A |f_n|^p d\mu$ gilt:
            \begin{enumerate}[label=\arabic*)]
              \item zu $\epsilon > 0 \ \exists \ \delta > 0$ mit $\lambda(A) < \epsilon \ \forall A$ messbar mit $\mu(A) < \delta$
              \item zu $\epsilon > 0 \ \exists \ E$ messbar mit $\mu(E) < \infty$ und $\lambda(X \setminus E) < \epsilon$ 
            \end{enumerate}
    \end{enumerate}
    Im Fall $p = 1$ heißt eine Folge mit 1) und 2) \textbf{gleichgradig integrierbar}.
  \end{theorem}
  \begin{proof}
    \item[] \underline{a) $\implies$ b):} \\
    Sei $f_n \to f$ in $L^p(\mu)$. FÜr $A$ messbar gilt: 
    $$ \left| ||f_n||_{L^p(A)} - ||f||_{L^p(A)} \right| \leq ||f_n-f||_{L^p(A)} \leq ||f_n-f||_{L^p} \to 0$$ 
    $\implies \lim\limits_{n\to\infty} ||f_n||_{L^p(A)} = ||f||_{L^p(A)}$ und damit $\lambda(A) = \int\limits_{A}|f|^p d\mu$ \\
    1) gilt dann nach Blatt 6, Aufgabe 2,3 \\
    (3) $\implies \lambda$ äußeres Maß, 2) $\implies$ Beh 1) \\
    Weiter ist mit Lemma IV.7 $\mu(E_\delta) \leq \frac{1}{\delta}\int |f|^p d\mu < \infty \text{für } E_\rho = \{ |f|^p \geq \delta \} $ \\
    $\implies \psi_{X\setminus E_\delta} |f|^p \leq \delta \to 0$ mit $\delta \to 0$, wobei $|f|^p$ integrierbare Majorante ist $\implies \lambda(X\setminus E_j) = \int\limits_{X\setminus E_j} |f|^p d\mu < \epsilon$ für $\delta$ klein genug. 
    \item[]\underline{b)$\implies$ a):}\\
    Zu $\epsilon > 0$ wähle $E$ wie in 2) sowie $\delta >0$ wie in 1). Da $\mu(E) < \infty$ existiert nach Egorov, Satz I.21 $A\subset E$ messbar mit $\mu(E\setminus A) < \delta$, sodass $f_n \to f$ gleichmäßig auf $A$. \\
    Hier wenden wir Satz I.21 auf das Maß $\mu_E = \mu|_{\script{D}(E)}$ an. 
	Die Menge $A$ ist dann $\mu_E$-messbar und damit auch $\mu$-messbar, denn $S \subset X$ und $E \mu$-messbar: 
	\begin{equation*}
	 \begin{split} \mu(S) &\geq \mu(S\cap E) + \mu(S\setminus E) \\
	 &\geq \mu(S\cap A) + \mu (S\cap E \setminus A) + \mu(S\setminus E) \\
	 &\geq \mu(S\cap A) + \mu(S\setminus A)
	 \end{split}
	\end{equation*}
	Nun gilt $|f_n -f|^p \leq \psi_A |f_n-f|^p + (\psi_{X\setminus E}+\psi_{E\setminus A}) 2^{p-1}(|f|^p+|f_n|^p)$ \\
	Lemma von Fatou impliziert für $B$ messbar $$\int\limits_B |f|^p d\mu \leq \liminf\limits_{n\to\infty} \int\limits_B |f_n|^p d\mu \leq \lambda(B)$$
	Mit $n\to\infty$ folgt aus 2) und 1) $$\limsup \int |f_n-f|^p d\mu \leq 2^p (\lambda(x\setminus E)+\lambda (E\setminus A)) < 2^{p+1}\epsilon$$
  \end{proof}

  