\chapter{Satz von Fubini}
  \begin{remark}[Prinzip von Cavalieri]
    Haben zwei Körper in jeder Höhe Schnitte von gleichem Flächeninhalt, so haben sie auch auch das gleiche Volumen.    
  \end{remark}

  \begin{example}
    \begin{enumerate}
      \item Volumen eines Kegels.\\
        Sei $B \subseteq \mathbb{R}^2$ ein Gebiet mit Flächeninhalt $|B| = A$.\\
        Betrachte $C = \{s(b,1) \ | \ b \in B, 0 \leq s \leq 1\}$\\
        \includegraphics[width=3.5cm]{img/VI_Bsp_1_Kegel.png}\\
        Für $k \in [0,1]$ ist der $k$-Schnitt von $C$ die Menge \\
        $C_k = \{a \in \mathbb{R}^2 \ | \ (a,b) \in C\} = \{h \cdot b \ | \ b \in B\} = kB$\\
        $|C_k| = k^2A$ Nach Cavalieri hängt $vol(C)$ nur von $A$ ab.\\
        Wir schreiben $vol(C) = V(A)$. Es gilt $V(kA) = kV(A)$ für $k \in \mathbb{N}$, betrachte dazu ... (siehe Aufschrieb)
        \item Betrachte in $\mathbb{R}^3 = \mathbb{R}^2 \times \mathbb{R}$ die Mengen und die Inhalte der Zugehörigen $z$-Schnitte:
        \begin{align*}
          &\text{Zylinder } Z=\{(x,y,z) \ | \ \sqrt{x^2 + y^2} \leq 1, 0 \leq z \leq 1\}, |Z_z| = \pi\\
          &\text{Kegel } C=\{(x,y,z) \ | \ \sqrt{x^2 + y^2} \leq z, 0 \leq z \leq 1\}, |C_z| = \pi z^2\\
          &\text{Halbkugel } H=\{(x,y,z) \ | \ \sqrt{x^2 + y^2} \leq \sqrt{1-z^2}, 0 \leq z \leq 1\}, |H_z| = \pi (1-z^2)\\
        \end{align*}
        $\implies |Z_z| = |C_z| + |H_z| \stackrel{Cavalieri}{\implies} vol(H) = vol(Z) - vol(C) = \pi - \frac{\pi}{3} = \frac{2}{3} \pi$\\
        $\implies vol(C) : vol(H) : vol(Z) = 1 : 2 : 3$ (Archimedes)
    \end{enumerate}
  \end{example}

  \begin{definition}
    Seien $\alpha, \beta$ äußere Maße auf $X,Y$. Das \textbf{Produktmaß} $\alpha \times \beta$ einer Menge $E \subseteq X \times Y$ ist
    \begin{align*}
      (\star) \ \ \alpha \times \beta (E) = inf\{\sum\limits_{j=1}^{\infty} \alpha(A_j) \beta(B_j) \ | \ A_j, B_j \text{ messbar}, E \subseteq \bigcup\limits_{j=1}^{\infty} A_j \times B_j\}
    \end{align*}
  \end{definition}

  \begin{lemma}
    $\alpha \times \beta$ ist ein äußeres Maß
  \end{lemma}
  \begin{proof}
    \item[$\cdot$] $\alpha \times \beta (\varnothing) \leq \alpha (\varnothing) \beta (\varnothing) = 0$ 
	\item[$\cdot$] Sei $E\subset \bigcup\limits_{i=1}^\infty E_i$ mit $E, E_i \subset X\times Y$. Wähle zu $\epsilon >0$ Überdeckungen $E_i \subset \bigcup\limits_{j=1}^\infty A_{i,j}\times B_{i,j}$ wie in $(\ast)$, sodass $\sum\limits_{j=1}^\infty \alpha (A_{i,k}) \beta(B_{i,j}) < \alpha \times \beta (E_i) + 2^{-i} \epsilon$ \\
	$\implies E\subset \bigcup\limits_{i,j=1}^\infty A_{i,j}\times B_{i,j}$\\
	$\implies \alpha \times \beta (E) \leq \sum\limits_{i,j=1}^\infty \alpha(A_{i,j})\beta(B_{i,j}) < \sum\limits_{i=1}^\infty \alpha \times \beta (E_i)+\epsilon$
  \end{proof}

  \begin{lemma}
    Sei $P = A \times B$. Eine Produktmenge, d.h. $A,B$ sind messbar bzgl. $\alpha$ bzw. $\beta$. Dann gilt $\alpha \times \beta(P) = \alpha(A) \beta(B)$ und $P$ ist $\alpha \times \beta$-messbar.
  \end{lemma}

  \begin{proof}
	$A,B$ messbar $\implies alpha\times\beta(P)\overset{(\ast)}{\leq}\alpha(A)\beta(B)$\\
	\item[]\underline{z.z:} $\alpha (A) \beta(B) \leq \alpha\times\beta(P)$\\
	Der $y$-Schnitt von $P$ ist $P_y = \{x\in X: (x,y) \in P \} = \begin{cases} A \text{, für } y\in B \\
	\varnothing \text{ sonst } \end{cases}$ \\
	Ist $P \subset \bigcup\limits_{j=1}^\infty P_j$ mit $P_j = A_j\times B_j$, so folgt $P_y \subset \bigcup\limits_{j=1}^\infty (P_j)_y$ und mit Fatou \\ $\alpha(A)\beta(B) = \int\limits_y \alpha(P_y) d\beta(y) \leq \int\limits_y \sum\limits_{j=1}^\infty \alpha((P_j)_y) d\beta(y) = \sum\limits_{j=1}^\infty \int\limits_y \alpha((P_j)_y) d\beta(y) = \sum\limits_{j=1}^\infty \alpha(A_j)\beta(B_j)$ \\
	$\implies \alpha(A)\beta(B) \leq \alpha\times\beta(P)$\\
	\item[]Messbarkeit: \\
	Sei $S\subset\bigcup\limits_{j=1}^\infty P_k$ mit $P_j = A_j\times B_j$ wie in $(\ast)$. \\
	Es gilt: 
	\begin{align*}
		(\ast\ast) && P_j \cap P = (A_j \cap A) \times (B_j \cap B) && P_j\setminus P = (A_j\cap A) \times (B_j\setminus B) \cup (A_j\setmiinus A) \times B_j 
	\end{align*}
	$\implies$ 
	\begin{equation*}
		\begin{split}
		\alpha\times\beta(P_j\cap P) + \alpha\times\beta (P_j\setminus P) &\leq \alpha(A_j \cap A) \beta(B_j\cap B) + \alpha(A_j\cap A)\beta(B_j\setminus B) + \alpha(A_j\setminus A) \beta(B_j) \\
	&= \alpha(A_j\cap A) \beta(B_j) + \alpha(A_j\setminus A)\beta(B_j) \\
	&= \alpha(A_j) \beta(B_j)
	\end{split}
	\end{equation*}
	$\implies \alpha\times\beta(S\cap P) + \alpha\times\beta(S\setminus P) \leq \sum\limits_{j=1}^\infty \alpha\times\beta(P_j\cap P) + \sum\limits_{j=1}^\infty \alpha\times\beta (P_j \setminus P) \leq \sum\limits_{j=1}^\infty \alpha(A_j)\beta(B_j)$\\
	$\overset{(\ast)}{\implies} \alpha\times\beta(S\cap P) + \alpha\times\beta(S\setminus P) \leq \alpha\times\beta(S)$
  \end{proof}

  \begin{theorem}[Cavalierisches Prinzip]
    Seien $\alpha$ und $\beta$ $\sigma$-endliche äußere Maße auf $X$ bzw. $Y$, und $D \subseteq X \times Y$ sei $\alpha \times \beta$-messbar. Dann ist $D_y = \{x \in X \ | \ (x,y) \in D\}$ $\alpha$-messbar für $\beta$-fast alle $y \in Y$. Die Funktion $y \mapsto \alpha(D_y)$ ist $\beta$-messbar und es gilt:
    \begin{align*}
      (\alpha \times \beta)(D) = \int\limits_Y \alpha(D_y) \ d\beta(y)
    \end{align*}
  \end{theorem}
  \sidenote{Vorlesung 17}{11.01.21}
  \begin{proof}
    \item[I] \underline{$(\alpha\times\beta)(D) < \infty$} \\
    Nach $(\ast)$ und Lemma VI.3 existiert zu $r\in\mathbb{N}$ eien Überdeckung $E^r = \bigcup\limits_{i=1}^\infty P_i^r$ von $D$ mit Produktmenge $P_i^r$, sodass 
   	$\sum\limits_{i=1}^\infty \alpha\times \beta (P_i^r) < \alpha\times\beta(D) + \frac{1}{r}$
   	\begin{itemize} 
   	\item[a)] $P_i^r$, $i\in\mathbb{N}$, sind disjunkt, sonst betrachte induktiv $P^r\setminus \bigcup\limits_{j=1}^{i-1}P_j^r$. Nach $(\ast\ast)$ ist dies eine disjunkte Vereinigung von Produktmengen.
   	\item[b)] $E^r\subset E^{r-1}$ $\forall r$, sonst gehe über zur disjunkten Überdeckung $P^r_i \cap P_j^{r-1}$ mit $i,j\in\mathbb{N}$. Nach $(\ast\ast)$ sind dies Produktmengen und \\
   	$\sum\limits_{i,j=1}^\infty \alpha\times\beta (P_i^r\cap P_j^{r=1}) = \sum\limits_{i=1}^\infty \alpha\times\beta(\bigcup\limits_{j=1}^\infty P_i^r\cap P_j^{r-1}) \leq \sum\limits_{i=1}^\infty\alpha\times\beta (P_i^r)<\alpha\times\beta (D) + \frac{1}{r}$ Mit $E:=\bigcap\limits_{r=1}^\infty E^r$ folgt: $D\subset \bigcap\limits_{r=1}^\infty E^r = E$ und $\alpha\times\beta(E) = \alpha\times\beta(D)$. Da $D$ messbar ist, gilt $\alpha\times\beta(E\setminus D) = \alpha\times\beta (E) - \alpha\times\beta(D) = 0$. 
   	\end{itemize} 
   	\item[]\underline{Wir wollen 3 Aussagen zeigen}
   	\begin{itemize}
   		\item[1)] $E_y = \{x\in X: (x,y)\in E \}$ ist $\alpha$-messbar $\forall y\in Y$
   		\item[2)] $f_k: Y \to [0,\infty]$, $y \mapsto \alpha(E_y)$ $\beta$-messbar
   		\item[3)] $\gamma (?) (E) := \int\limits_Y f_k d\beta = \alpha\times\beta (E)$
   	\end{itemize}
   Sei $\epsilon$ das System aller $\alpha\times\beta$-messbaren Mengen, welche 1), 2) und 3) erfüllen. Für Produktmenten $A\times B$ gilt: \\
   $(A\times B)_y = A$ falls $y=B$ und $(A\times B)_y = \varnothing$, falls $y\notin B$ $\implies A\times B \in \epsilon$, denn $f_{A\times B} = \alpha (A)\psi_B$ und $\gamma(A\times B) = \alpha(A) \beta(B) = \alpha\times\beta(A\times B)$. \\
   Jetzt betrachte disjunkte Vereinigung $E = \bigcup\limits_{i=1}^\infty E_i$ mit $E_i \in \epsilon$ $\forall i\in \mathbb{N}$. \\
   $\implies E_y = \bigcup\limits_{i=1}^\infty (E_i)_y$ $\alpha$-messbar \\
   $f_E = \sum\limits_{i=1}f_{E_i}$ ist $\beta$-messbar und \\
   $\gamma(E) = \int\limits_Y \sum\limits_{i=1}^\infty f_{E_i} d\beta \overset{\text{Satz von der Mon. Konv.}}{=} \sum\limits_{i=1}^\infty \int\limits_Y f_{E_i} d\beta = \sum\limits_{i=1}^\infty \alpha\times\beta(E_i) = \alpha\times\beta(E)$ \\
   Schließlich sei $E^1 \supset E^2 \supset ...$ mit $E^r \in \epsilon$ und $\alpha\times\beta(E^1) < \infty$. \\
   Für $E = \bigcap\limits_{r=1}^\infty E^r$ ist $E_y = \bigcap_{r=1}^\infty (E^r)_y$ $\alpha$-messbar $\forall y\in Y$. Lemma II.8 impliziert \\
   $f_E(y) = \alpha (E_y) = \lim\limits_{r\to\infty} \alpha((E^r)_y) = \lim\limits_{r\to\infty} f_{E^r}(y)$ für $\beta$-fast alle $y\in Y$. \\
   $\implies f_E$ ist $\beta$-messbar. \\
   Zuletzt folgt aus Satz von Lebesgue wegen $f_{E^r} \leq f_{E^1}$ \\
   $\gamma(E) = \int\limits_Y f_E d\beta = \lim\limits_{r\to\infty} \int\limits_Y f_{E^r} d\beta = \lim\limits_{r\to\infty}\alpha\times\beta(E^r) = \alpha\times\beta(E)$ \\
   $\implies$ Für $E$ wie oben gelten somit die Aussagen 1), 2), und 3). \\
   Jetzt wende das Argument auf eine $\alpha\times\beta$ Nullmenge $N$ statt $D$ an. Erhalte $C\supset N$ mit $C\in\epsilon$ und $\alpha\times\beta(C) = 0$. Also \\
   $0 = \alpha\times\beta(C) = \int\limits_Y \alpha(C_y)d\beta(y) \implies \alpha(N_y) \leq \alpha(C_y) = 0$ für fast alle $y\in Y$. \\
   Wähle $N=E\setminus D$. Dann folgt für $\beta$-fast alle $y\in Y$: \\
   $D_y = E_y \setminus N_y$ ist $\alpha$-messbar und es gilt: $f_D(y) = f_E(y)$ ($\implies f_D$ ist $\beta$-messbar)\\
   $\int\limits_Y f_D d\beta = \int\limits_Y f_E d\beta = \alpha\times\beta(E) = \alpha\times\beta(D)$. 
   \item[II] Sei $D\subset X\times Y$ nur messbar.\\
   Nach Vor. gilt $X=\bigcup\limits_{n=1}^\infty A_n$, $Y=\bigcup\limits_{n=1}^\infty B_n$ mit $A_n$, $B_n$ messbar und $\alpha(A_n)$, $\beta(B_n) < \infty$. \\
   O.E. $A_n$, $B_n$ aufsteigend. $D_n = D\cap (A_n\times B_n)$ ist $\alpha\times\beta$-messbar mit $(\alpha\times\beta)(D_n) < \infty$ \\
   $\overset{\text{I}}{\implies}$ Dann ist $D_y = \bigcup\limits_{n=1}^\infty (D_n)_y$ $\alpha$-messbar für $\beta$-fast alle y. Es gilt $f_{D_1} \leq f_{D_2} \leq ...$ und $f_D(y) = \alpha(D_y) = \lim\limits_{n\to\infty}(\alpha(D_n)_y) = \lim\limits_{n\to\infty}f_{D_n}(y) \\
   \implies f_D$ $\beta$-messbar und $\int\limits_Y f_D d\beta \overset{\text{mon.konv.}}{=} \lim\limits_{n\to\infty}\int\limits_Y f_{D_n}d\beta = \lim\limits_{n\to\infty} \alpha\times\beta(D_n) = \alpha\times\beta(D)$
  \end{proof}

  \begin{remark}
    Die Rollen von $\alpha$ und $\beta$ können vertauscht werden, d.h. man betrachtet das $\beta$-Maß des $X$-Schnittes $D_x 0 \{y \in Y \ | \ (x,y) \in D\}$ und integriert bzgl. $\alpha$
    \begin{align*}
      \implies \alpha \times \beta (D) = \int\limits_Y \alpha(D_y) \ d\beta(y) = \int\limits_X \beta(D_x) \ d\alpha(x)
    \end{align*}
  \end{remark}

  \begin{example}
    Man kann nicht auf die $\sigma$-Endlichkeit verzichten.
    \begin{align*}
      D := \{(x,y) \in [0,1] \times [0,1] \ | \ x=y\} \subseteq \mathbb{R} \times [0,1]\\
      \int\limits_{\mathbb{R}} card(D_x) d\lambda^1(x) = 1 \neq 0 = \int\limits_{[0,1]} \lambda^1(D_y) \ d card(y)
    \end{align*}
    Mit $I_k = [\frac{k-1}{n}, \frac{k}{n}]$ gilt $D = \bigcap\limits_{n=1}^{\infty}( \bigcup\limits_{k=1}^{\infty} I_k \times I_k) \implies D$ ist messbar bzgl. $\lambda^1 \times card$
  \end{example}

  \begin{lemma}
    Es gilt $\lambda^n = \lambda^k \times \lambda^m$ für $k + m = n$
  \end{lemma}

  \begin{proof}
    Quader im $\mathbb{R}^n$ sind Produkte $P\times Q$ von Quadern im $\mathbb{R}^k$ und $\mathbb{R}^m$ und es gilt $vol^n(P\times Q) = vol^k(P)vol^m(Q)$. Somit ist $\lambda^n(E) = \inf \{\sum\limits_{j=1}^\infty vol^k(P_j)vol^m(Q_j):$ $ P_j, Q_j$ Quader, $E\subset \bigcup\limits_{j=1}^\infty P_j\times Q_j \}$ \\
    Es folgt direkt $\lambda^k \times \lambda^m \leq \lambda^n$. \\
    \item[\underline{z.z}] $\lambda^n \leq \lambda^k\times\lambda^m$ \\
    Reicht dies für Produktmengen zu zeigen, d.h. $\lambda^n(A\times B) \leq \lambda^k(A) \lambda^m(B)$, denn daraus folgt für $E\subset \mathbb{R}^n$ bel.: 
    $$ \lambda^n(E) \leq \inf\{\sum\limits_{j=1}^\infty \lambda^n(A_j\times B_j): A_j, B_j \text{ messbar }, E\subset \bigcup\limits_{j=1}^n A_j\times B_j \} \leq (\lambda^k\times\lambda^m)(E)$$ 
    Betrachte nun $A\times B$ mit $\lambda^k(A)$, $\lambda^m(B) <\infty$. Zu $\epsilon >0$ existiert Quader $P_i$, $Q_i$ mit $A\subset \bigcup\limits_{i=1}^\infty P_i$, $B\subset \bigcup\limits_{j=1}^\infty Q_j$, sodass $\sum\limits_{i=1}^\infty vol^k(P_i) < \lambda^k(A)+\epsilon$, $\sum\limits_{j=1}^\infty vol^m(Q_j) < \lambda^m(B)+\epsilon$. \\
    $\implies A\times B \subset \bigcup\limits_{i,j=1}^\infty (P_i\times Q_j) \\
    \implies \lambda^n(A\times B) \leq \sum\limits_{i,j=1}^\infty vol^k(P_i) vol^m(Q_j) < (\lambda^k(A) + \epsilon) (\lambda^m(B) + \epsilon) \overset{\epsilon\to 0}{\to} \lambda^k(A)\lambda^m(B)$. \\
    Für $\lambda^k(A)\lambda^m(B) = \infty$ ist nichts zu zeigen. \\
    Bleibt der Fall $\lambda^k(A) = \infty$, $\lambda^m(B) = 0$ (oder umgekehrt).\\
    Mit $A_l = \{x\in A:$ $l-1\leq ||x|| \leq l \}$ folgt \\
    $$\lambda^n(A\times B) \leq \sum\limits_{l=1}^\infty \lambda^n(A_l\times B) \leq \sum\limits_{l=1}^\infty \lambda^k(A_l)\lambda^m(B) = 0$$
  \end{proof}
  \newpage 
  
  \sidenote{Vorlesung 18}{15.01.21}
  \begin{example}
    \begin{enumerate}
      \item[]
      \item Volumen $\alpha_n$ der Kugel $B=\{z \in \mathbb{R}^n \ | \ ||z|| < 1\}.$\\
        Für $y \in [-1,1)$ ist $B_y = \{x \in \mathbb{R}^{n-1} \ | \ ||x|| < |1-y^2|^{1/2}\}$\\
        \includegraphics[width=3.5cm]{img/VI_Bsp_3_Kreis.png}
        \begin{align*}
          \alpha_n &=  \int\limits_{-1}^1 \lambda^{n-1}(B_y) \ dy = \alpha_{n-1} \int\limits_{-1}^1 (1-y^2)^{\frac{n-1}{n}} \ dy\\
          &\stackrel{y=\cos \theta}{=} \alpha_{n-1} A_n \text{, mit } A_n = \int\limits_0^{\pi} \sin^n\theta \ d\theta\\
          &\stackrel{part. Int.}{\implies} A_n = \frac{n-1}{n} A_{n-2} \ \forall \ n \geq 2 \text{, dabei sind } A_0 = \pi, A_1 = 2\\
          \implies A_{2k} &= \frac{2k-1}{2k} \cdot ... \cdot \frac{1}{2} A_0 = \pi \prod\limits_{j=1}^k \frac{2j}{2j+1}\\
          A_{2k+1} &= \frac{2k}{2k+1} \cdot ... \cdot \frac{2}{3} A_1 =2 \prod\limits_{j=1}^k \frac{2j}{2j+1}\\
          \implies A_{2k+1} A_{2k} &= \frac{2\pi}{2k+1} \text{ bzw. } A_{2k} A_{2k-1} = \frac{\pi}{k}\\
          \implies \alpha_{2k} &= (A_{2k}A_{2k-1}) ... (A_3 A_2) \alpha_0 = \frac{\pi^k}{k!}\\
          \alpha_{2k+1} &= (A_{2k}A_{2k-1}) ... (A_3 A_2) \alpha_1 = \frac{\pi^k}{(k+\frac{1}{2})(k-\frac{1}{2})...\frac{1}{2}}
        \end{align*}
        Bem: $\alpha_k \to 0$ mit $k \to \infty$
      \item Für $A \subseteq \mathbb{R}^n$ sei $K(A) = \{y (x,1) \in \mathbb{R}^n \times \mathbb{R} = \mathbb{R}^{n-1} \ | \ 0<y<1, x\in A\}$\\
        Beh: $A$ messbar bzgl. $\lambda^n$ $\implies$ $K(A)$ $\lambda^{n+1}$-messbar und $\lambda^{n+1}(K(A)) = \frac{1}{n+1}\lambda^n(A)$\\
        (siehe Aufschrieb)
    \end{enumerate}
  \end{example}

  \begin{definition}
    Eine Funktion $f: X \to [-\infty, \infty]$ heißt $\bm{\sigma}$\textbf{-endlich} bzgl. des äußeren Maßes $\mu$, falls gilt:
    $$f \text{ ist } \mu \text{-messbar und } \{f \neq 0\} \text{ ist } \sigma \text{-endlich}$$
  \end{definition}

  \begin{theorem}[Fubini]
    Seien $\alpha, \beta$ äußere Maße auf $X$ bzw. $Y$ und $f: X \times Y \to \bar{\mathbb{R}}$ sei $\sigma$-endlich bzgl. $\alpha \times \beta$. Ist das Integral $\int f \ d(\alpha \times \beta)$ definiert, so gilt:
    \begin{enumerate}
      \item Für $\beta$-fast alle $y \in Y$ ist $f(\cdot, y) \alpha$-messbar, und $\int\limits_X f(x,y) \ d \alpha(x)$ existiert.
      \item $y \mapsto \int\limits_X f(x,y) \ d\alpha(x)$ ist $\beta$-messbar und $\int\limits_Y \int\limits_X f(x,y) \ d\alpha(x) \ d\beta(y)$ existiert.
      \item $\int\limits_{X\times Y} f \ d(\alpha \times \beta) = \int\limits_Y \int\limits_X f(x,y) \ d\alpha(x) \ d\beta(y)$
    \end{enumerate}
    Der Satz gilt auch mit vertauschten Reihenfolgen der Integrationen, also folgt:
    \begin{align*}
      \int\limits_{X\times Y} f \ d(\alpha \times \beta) = \int\limits_Y \int\limits_X f(x,y) \ d\alpha(x) \ d\beta(y) = \int\limits_X \int\limits_Y f(x,y) \ d\beta(y) \ d\alpha(x)
    \end{align*}
    Zusatz: Ist $f: X \times Y \to \bar{\mathbb{R}} \ \sigma$-endlich und $\int\limits_Y \int\limits_X |f(x,y)| \ d\alpha(x) \ d\beta(y) < \infty$, so ist $f$ integrierbar bzgl. $\alpha \times \beta$ und der Satz damit anwendbar.
  \end{theorem}

  \begin{proof}
    Für $f = \psi_E$ mit $E\subset X\times Y$ $\sigma$-endlich gelten 1)-3) nach Satz VI.4
    \begin{itemize}
    	\item[$\cdot$] Für $\beta$-fast alle $y\in Y$ ist $f(\cdot, y) = \psi_{E_y}$ $\alpha$-messbar mit $\int\limits_X f(x,y) d\alpha(x) = \alpha(E_y)$
    	\item[$\cdot$] $y \mapsto \alpha(E_y)$ ist $\beta$-messbar mit $\int\limits_Y \alpha(E_y) d\beta(y)$ 
    	\item[$\cdot$] $\int\limits_{X\times Y} f d(\alpha\times\beta) = (\alpha\times\beta)(E) = \int\limits_Y \alpha(E_y)d\beta(y) = \int\limits_Y \int\limits_X f(x,y) d\alpha(x) d\beta(y)$
    \end{itemize}
	Sei jetzt $f\leq 0$. Satz IV.9 $\implies \exists \alpha\times\beta$-Treppenfunktion $0 \leq f_1 \leq f_2 \leq ...$ mit $f_k(x,y) \to f(x,y)$ auf $X\times Y$. Die $f_k$ sind ebenfalls $\sigma$-endlich, und es gilt: 
	\begin{itemize}
		\item[$\cdot$] $f(\cdot, y)$ ist monotoner Grenzwert der $f_k(\cdot, y)$. Für $\beta$-fast alle $y\in Y$ ist damit $f(\cdot, y)$ $\alpha$-messbar und $\int\limits_X f(\cdot, y)d\alpha = \lim\limits_{k\to\infty} \int\limits_x f_k(\cdot, y) d\alpha$ (Satz IV.10)
		\item[$\cdot$] $y \mapsto \int\limits_X f(\cdot, y) d\alpha$ ist monotoner Grenzwert von $y\mapsto \int\limits_X f_k(\cdot, y) d\alpha$ für $\beta$-fast alle $y\in Y$\\ $\implies y\mapsto \int\limits_X f(\cdot, y) d\alpha$ ist $\beta$-messbar und $\int\limits_Y \int\limits_X f(x,y)d\alpha(x) d\beta(x) = \lim\limits_{k\to\infty}\int\limits_Y \int\limits_X f_k(x,y) d\alpha(x) d\beta(y)$
		\item[$\cdot$] $\int\limits_{X\times Y} f d(\alpha\times\beta) = \lim\limits_{k\to\infty}\int\limits_{X\times Y} f_k d(\alpha\times\beta) = \lim\limits_{k\to\infty}\int\limits_Y\int\limits_X f_k (x,y) d\alpha(x) d\beta(y) = \int\limits_Y \int\limits_X f(x,y) d\alpha(x) d\beta(y) $
	\end{itemize}
	$\implies$ Fubini für $f\geq 0$ \\
	Sei $f$ nun $\sigma$-endlich und $\int f^- d(\alpha\times\beta) < \infty$. \\
	$\implies f^\pm$ sind auch $\sigma$-endlich und $\int\limits_Y \int\limits_X f^-(x,y) d\alpha(x) d\beta(y) = \int\limits-{X\times Y} f^- d(\alpha\times\beta) y\infty$ \\
	$\implies$ Für $\beta$-fast alle $y\in Y$ gilt $\int\limits_X f^-(x,y) d\alpha(x) < \infty$ und $f^-(x,y) < \infty$ für $\alpha$-fast alle $x\in X$. 
	\begin{itemize}
		\item[$\cdot$] Für $\beta$-fast alle $y\in Y$ ist $f(\cdot, y) = f^+(\cdot, y) - f^-(\cdot, y)$ $\alpha$-messbar, mit Integral $\int\limits_X f(\cdot, y) d\alpha = \int\limits_X f^+(\cdot,y)d\alpha - \int\limits_X f^-(\cdot,y)d\alpha$
		\item[$\cdot$] $y\mapsto \int\limits_X f(\cdot,y) d\alpha$ ist Differenz zweier messbarer Funktionen, also messbar und ihr Integral existiert, denn $s\mapsto s^-$ fallend, gilt \\
		$\int\limits_Y \left(\int\limits_X f(x,y) d\alpha(x) \right)^- d\beta(y) \leq \int\limits_Y \left(-\int\limits_X f^-(x,y) d\alpha(x) \right)^- d\beta(y) \\= \int\limits_Y \int\limits_X f^-(x,y) d\alpha(x) d\beta(y) < \infty$
		\item[$\cdot$] \begin{equation*}
			\begin{split}
				\int\limits_{X\times Y} f d(\alpha\times\beta) &= \int\limits_{X\times Y} f^+ d(\alpha\times\beta) - \int\limits_{X\times Y} f^- d(\alpha\times\beta) \\
				&= \int\limits_Y \int\limits_X f^+(x,y) d\alpha(x) d\beta(y) - \int\limits_Y \int\limits_X f^-(x,y) d\alpha(x) d\beta(y) \\
				&= \int\limits_Y \int\limits_X \left( f^+(x,y) -f^-(x,y)\right) d\alpha(x) d\beta(y) \\
				&= \int\limits_Y \int\limits_X f(x,y) d\alpha(x) d\beta(y)
			\end{split}
		\end{equation*}
		$\implies$ Beh.
	\end{itemize}
	Zusatz: Folgt durch Anwendung auf $|f|$, dann gilt: $$ \int\limits_{X\times Y} |f| d(\alpha\times\beta) = \int\limits_Y \int\limits_X |f(x,y)|d\alpha(x) d\beta(y) < \infty$$
	$\implies$ Integral von $f$ bzgl $\alpha\times\beta$ ist definiert.
  \end{proof}

  \begin{example}
    \begin{enumerate}
      \item[]
      \item 
          $\begin{rcases}
            \int\limits_{-1}^1 \int\limits_{-1}^1 \frac{x^2-y^2}{(x^2+y^2)^2} \ dy \ dx = \pi\\
            \int\limits_{-1}^1 \int\limits_{-1}^1 \frac{x^2-y^2}{(x^2+y^2)^2} \ dx \ dy = -\pi
          \end{rcases} \text{ denn } \frac{x^2-y^2}{(x^2+y^2)^2} = \frac{\partial}{\partial x} \frac{\partial}{\partial y} \arctan(\frac{x}{y}) \text{ für } y \neq 0$\\
          Fubini $\implies$ Integral bzgl. $\lambda^2 = \lambda^1 \times \lambda^1$ ex nicht!
      \item
        \begin{align*}
          \int\limits_{-1}^1 \int\limits_{-1}^1 \frac{xy}{(x^2+y^2)^2} \ dx \ dy = 0 = \int\limits_{-1}^1 \int\limits_{-1}^1 \frac{xy}{(x^2+y^2)^2} \ dy \ dx
        \end{align*}
        Aber das $\lambda^2$-Integral über $[-1,1)^2$ ex. nicht, da
        \begin{align*}
          \int\limits_{[0,1)^2} \frac{xy}{(x^2+y^2)^2} \ d\lambda^2(x,y) = \int\limits_0^1 \int\limits_0^1 \frac{xy}{(x^2+y^2)^2} \ dx \ dy = \frac{1}{2} \int\limits_0^1 (\frac{1}{y} - \frac{y}{1+y^2}) \ dy = \infty
        \end{align*}
    \end{enumerate}
  \end{example}

  \begin{example}
    $\mu$ äußeres Maß auf $X$ und $f: X \to [0, \infty]$ sei $\sigma$-endlich bzgl. $\mu$. Ist $\script{C}:[0,\infty] \to [0, \infty]$ stetig mit $\script{C}(0) = 0$, sowie auf $(0, \infty)$ stetig differenzierbar mit $\script{C}'(t)\geq0$, so gilt:
    \begin{align*}
      \int\limits_X \script{C}(f(x)) \ d\mu(x) = \int\limits_0^{\infty} \script{C}'(t) \mu(\{f > t\}) \ dt
    \end{align*}
    (Begründung siehe Aufschrieb)
  \end{example}

  \sidenote{Vorlesung 19}{18.01.21}
  \begin{theorem}
    $\Omega \subseteq \mathbb{R}^n$ offen. Für $f \in C_C^1(\Omega)$ und $g \in C^1(\Omega)$ gilt:
    $$\int\limits_{\Omega}(\partial_j f)g dx = -\int\limits_{\Omega} f (\partial_j g) dx \ \ \ \ \forall \ 1 \leq j \leq n \ (dx \ \hat{=} \ d \lambda^n)$$
  \end{theorem}
  \begin{proof}
    Es reicht die Aussage $(\ast)$ zu zeigen: 
    \begin{align*}
    	(\ast) && \int\limits_{\mathbb{R}^n} \partial_j f dx = 0 \text{   } \forall f\in C_c^1 (\mathbb{R}^n)
    \end{align*}
	Denn setzen wir $fg$ durch $0$ zu einer Funktion $\phi \in C^1_c(\mathbb{R}^n)$ fort, so folgt $$ 0 = \int\limits_{\mathbb{R}^n} \partial_j \phi dx = \int\limits_\Omega \partial_j (fg) dx = \int\limits_\Omega (\partial_j f) g dx + \int\limits_\Omega f \partial_j g dx$$
	Fubini für $f\in C^1_c(\mathbb{R}^n)$ liefert mit $x = (x', x_n)\in \mathbb{R}^{n-1}\times \mathbb{R}$ 
	\begin{equation*}
		\begin{split}
			\int\limits_{\mathbb{R}^n} \partial_j f(x) dx &= \int\limits_{\mathbb{R}} \int\limits_{\mathbb{R}^{n-1}} \partial_j f(x',x_n) dx' dx_n \\
			&= \int\limits_{\mathbb{R}^{n-1}} \int\limits_{\mathbb{R}} \partial_j f(x',x_n) dx_n dx'
		\end{split}
	\end{equation*}
	Für $j=n$ ist das letzte Integral $0$ nach Hauptsatz. Für $1 \leq j \leq n-1$ verschwindet das mittlere Integral nach Induktion. 
  \end{proof}

  \begin{remark}
    \begin{enumerate}
      \item[]
      \item partielle Integration wird oft mit $\triangledown$ und $div$ formuliert:\\
        $f \in C_C^1(\Omega), X \in C^1(\Omega, \mathbb{R}^n)$\\
        $\int\limits_{\Omega} <\triangledown f, x> dx = -\int\limits_{\Omega} f (div X) dx$\\
        $(<\triangledown f, X> = \sum\limits_{i=1}^n \partial_i f X_i \ , \ f div X = \sum\limits_{i=1}^n f \partial_i X_i)$
      \item Der Satz von Fubini gilt auch für kartesische Produkte mit endlich vielen (statt nur zwei) Faktoren. Man zeige analog zu Lemma VI.5, dass in einem endlichen Produkt von Maßen beliebig Klammern gesetzt oder weggelassen werden können. Fubini wird dann per Induktion über die Anzahl der Faktoren bewiesen.
    \end{enumerate}
  \end{remark}
