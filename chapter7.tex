\chapter{Der Transformationssatz}
  \begin{definition}
    Eine Abbildung $\Phi: \script{U} \to \script{V} \ ,\script{U}, \script{V} \subseteq \mathbb{R}^n$ offen, heißt $C^1$-Diffeomorphismus, falls $\Phi$ bijektiv ist und $\Phi, \Phi^{-1}$ stetig differenzierbar sind.
  \end{definition}

  \begin{example}[Polarkoordinaten in $\mathbb{R}^n$]
    $\Phi: (0, \infty) \times (0, 2\pi) = \script{U} \to \script{V} = \mathbb{R}^2 \setminus \{(x,0) \ | \ x \geq 0\}$\\
    \\
    $\Phi(r, \script{C}) = (r \cos(\script{C}), r \sin(\script{C}))$\\
    $\Phi^{-1}(x,y) = \begin{cases}
      (r, \arccos(\frac{x}{r})) & \text{, falls } y \geq 0\\
      (r, 2\pi - \arccos(\frac{x}{r})) & \text{, falls } y < 0
    \end{cases}\\
    r = \sqrt{x^2 + y^2}$\\
    \\
    Für $x < 0$ filt alternativ $\Phi^{-1}(x,y) = (r, \frac{\pi}{2} + \arccos(\frac{x}{r}))\\
    \implies \Phi^{-1}$ glatt auf ganz $\script{V} \implies \Phi^{C^1}$ Diffeomorphismus.
    \begin{center}
      \includegraphics[height=2cm]{img/VII_Bsp_1_Polarkoordinaten.png}
    \end{center}
  \end{example}

  \begin{remark}[Notation]
    $x \in \mathbb{R}^n, \delta >0$\\
    $Q(x, \delta) = \{y \in \mathbb{R}^n \ | \ ||y-x||_{\infty} \leq \delta\}, ||x||_{\infty} = \max\limits_{1 \leq k \leq n} | x_k |$
    \begin{center}
      \includegraphics[height=1cm]{img/VII_Notation_1_Qx1.png}
    \end{center}
  \end{remark}

  \begin{lemma}
    Sei $\script{U} \subseteq \mathbb{R}^n$ offen, $x_0 \in \script{U}$ und $\Phi: \script{U} \to \mathbb{R}^n$ mit $D\Phi(x_0) \in GL_n(\mathbb{R})$. Gegeben sei eine Folte $Q_j = Q(x_j, \phi_j)\subseteq \script{U}$ mit $\phi_j \to 0$ und $x_0 \in Q_j \ \forall \ j \in \mathbb{N}$. Dann gilt:
    $$\limsup\limits_{j \to \infty} \frac{\lambda^n(\Phi(Q_j))}{\lambda^n(Q_j)} \leq |det D\Phi(x_0)|$$
  \end{lemma}
  \begin{proof}
    O.E. $x_0 = 0$, $\phi(0) = 0$ (sonst betrachte $\tilde{\phi}(x) = \phi(x+x_0) - \phi (x_0)$)
    \item[]\underline{I. $D\phi (0) = \text{Id}$} \\
    $\overset{\phi \text{ diff in } 0}{\implies} 0 = \lim\limits_{x\to 0} \frac{||\phi(x) - (\phi (0) + D\phi (0) x) ||_\infty}{||x||_\infty} = \lim\limits_{x\to 0} \frac{|| \phi(x) - x ||_\infty}{||x||_\infty}$ \\
    (wegen $||x||_\infty \leq ||x|| \leq \sqrt{n} ||x||_\infty$ darf die $|| \cdot ||_\infty$ benutzt werden). \\
    Sei nun $\epsilon > 0$ für $x\in Q_j$ gilt: 
    $$ ||x||_\infty \leq ||x-x_j||_\infty + ||x_j-0||_\infty \leq 2 \phi_j$$ 
    $\implies$ Für $j$ hinreichend groß gilt: \\
    $$||\phi (x) - x ||_\infty \leq \epsilon ||x||_\infty \leq 2\epsilon \phi_j \text{   } \forall x\in Q_j$$
    $\implies ||\phi(x) - \phi(x_j)||_\infty \leq ||\phi(x) - x||_\infty + ||x-x_j||_\infty + ||\phi(x_j) - x_j||_\infty \leq (1+4\epsilon)\phi_j$ $\forall x\in Q_j$ \\
    $\implies \phi (Q_j) \subset Q(\phi(x_j), (1+4\epsilon)\phi_j)$ \\
    $\implies \frac{\lambda^n(\phi(Q_j))}{\lambda^n (Q_j)} \leq (1+4\epsilon)^n$ \\
    Mit $j\to\infty$ und $\epsilon\to\infty$ folgt die Behauptung im Fall I. 
    \item[] \underline{II $D\phi(0) \in GL_n(\mathbb{R})$} \\
    Sei $S = D\phi(0)$ und $\phi_0 := S^{-1} \circ \phi$, also $D\phi_0(0) = \text{Id}$.
    \begin{align*}
    \limsup\limits_{j\to\infty} \frac{\lambda^n(\phi(Q_j))}{\lambda^n(Q_j)} &= \limsup\limits_{j\to\infty} \frac{\lambda^n(S(\phi_0(Q_j)))}{\lambda^n(Q_j)} \\
    &= |\det{S}| \limsup\limits_{j\to\infty} \frac{\lambda^n(\phi_0(Q_j))}{\lambda^n(Q_j)} \\
    &\leq |\det{S}| = \det{D\phi(0)}
    \end{align*}
  \end{proof}
  \begin{theorem}[Transformationsformel]
    $\script{U}, \script{V} \subseteq\mathbb{R}^n$ offen, $\Phi:\script{U} \to \script{V} \ C^1$-Diffeomorphismus. Ist $A \subseteq \script{U}$ $\lambda^n$-messbar, so ist auch $\Phi(A)$ $\lambda^n$-messbar und es gilt:
    \begin{enumerate}
      \item $\lambda^n(\Phi(A)) = \int\limits_A | \det D\Phi(x) | dx$
    \end{enumerate}
    Weiter gilt für jede $\lambda^n$-messbare Funktion $f: \script{V} \to \bar{\mathbb{R}}$
    \begin{enumerate}[resume]
      \item $\int\limits_{\script{V}} f(y) dy = \int\limits_{\script{U}} f(\Phi(x)) \ |\det D\Phi(x)| \ dx \ \ (dy \ \hat{=} \ d\lambda^n(y))$
    \end{enumerate}
    falls eines der Integrale definiert ist.
  \end{theorem}

  \begin{proof}
    Betrachte auf $\sigma$-Algebra $\script{A}$ der $\lambda^n$-messbaren Mengen $A\subset U$ die Maße
    \begin{align*}
    	\lambda(A) = \int\limits_A |\det{D\phi}|d\lambda^n && \mu(A) = \lambda^n(\phi(A))	
    \end{align*}
	Def. Bildmaß $\rightarrow \mu(A) = \psi(\lambda^n)(A)$ mit $\psi = \phi^{-1}$. \\
	Für $\lambda^n$-messbares $A\subset U$ ist $\psi^{-1}(A) = \phi(A)$ $\lambda^n$-messbar	(Satz III.13) und damit ist $A$ auch $\mu$-messbar (s.Blatt 2, A4). \\
	Die Einschränkung des äußeren Maßes $\mu$ auf $\script{A}$ ist somit ein Maß. \\
	Weiter ist $\lambda$ das Lebesguemaß mit Dichte $|\det{D\phi}|$ (s.Blatt 6, A3), und damit folgt die Maßeigenschaft von $\lambda$ aus dieser Aufgabe. \\
	Weiter folgt aus Blatt 6, A3 bzw. Satz III.13 $(\ast)$ $N\subset U$ mit $\lambda^n(N)= 0 \implies \lambda(N) = \mu(N) = 0$
	\item[]\underline{z.z: $\lambda \geq \mu$} \\
	O.E. sei $\lambda(U) < \infty$, $\mu(U) < \infty$. Sonst schöpfe $U$ durch offene, relativ kompakte Mengen aus. \\
	Satz III.7 $\implies$ Jede $\lambda^n$-messbare Menge $A$ ist von der Form $A = E \setminus N$ mit $\lambda^n(N) = 0$ und $E = \bigcap\limits_{i=1}^\infty U_i$ mit offenen $U_1 \supset U_2 \supset ...$ \\
	Satz I.7 (ii) $\implies$ Reicht aus $\lambda \geq \mu$ auf allen offenen Mengen nachzuweisen. \\
	Lemma III.3 $\implies$ Jede offene menge ist abzählbare Vereinigung von kompakten, achsenparallelen Würfeln in $U$ mit paarweise disjunktem Inneren. \\
	Damit bleibt zu zeigen: $$ \lambda(Q_0) \geq \mu(Q_0) \text{   } \forall Q_0 = Q(r_0, \rho_0)\subset U$$
	Angenommen es ist $\lambda(Q_0) < \mu(Q_0)$ für ein $Q_0 = Q(r_0, \rho_0)$. Wähle $\theta \in [0,1]$ mit $\lambda(Q_0 < \theta \mu(Q_0))$ und zerlege $Q_0$ durch Halbierung der Kanten in $2^n$ kompakte Teilwürfel $Q_{0,1}, Q_{0,2},...,Q_{0,2^n}$. Wähle $\lambda(Q_{0,1}) \geq \theta \mu(Q_{0,1})$ $\forall i \implies \lambda(Q_0) = \sum\limits_{i=1}^{2^n} \lambda{Q_{0,1}} \geq \theta \sum\limits_{i=1}^{2^n} \mu(Q_{0,1}) = \theta \mu(Q_0)$ (Widerspruch !!) \\
	(Ränder sind Nullmengen bzgl $\lambda, \mu$ nach $(\ast)$) \\
	$\implies$ Unter $Q_{0,1}$ ex. einer (nennen wir $Q_1$) mit $\lambda(Q_1) < \theta \mu(Q_i)$ \\
	Bestimme damit induktiv eine Schachtelung $Q_0 \supset Q_1 \supset ...$ mit $(\ast\ast)$ $\lambda(Q_j) < \theta \mu(Q_j)$ $\forall j\in\mathbb{N}$ \\
	Es gilt $\bigcap\limits_{j=1}^\infty Q_j = \{x_0\}$ mit $x_0 \in U$ \\
	Da $\det{D\phi}$ stetig folgt mit $j\to\infty$ 
	$$ \left| \frac{\lambda(Q_j)}{\lambda^n(Q_j)} - |\det{D\phi(x_0)}|\right| = \frac{1}{\lambda^n(Q_j)} \left| \int\limits_{Q_j} (|\det{D\phi(x)}|-|\det{D\phi(x_0)})dx\right| \to 0 \text{ mit } j\to \infty$$
	Zusammen mit Lemma VII.2 folgt: 
	$$ \liminf\limits_{j\to\infty} \frac{\lambda(Q_j)}{\mu(Q_j)} = \liminf\limits_{j\to\infty} \left( \frac{\lambda(Q_j)}{\lambda^n(Q_j)} \cdot \frac{\lambda^n(Q_j)}{\mu(Q_j)}\right) \geq 1 $$
	Widerspruch zu $(\ast\ast)$ \\
	$\implies \lambda\geq\mu$ auf $\script{A}$
	Jetzt betrachte $f: V\to \mathbb{\bar{R}}$. Es gilt $\{f \circ \phi < s \} = \phi^{-1}(\{f < s \}) \implies $ Mit $f$ ist auch $f\circ \phi$ $\lambda^n$-messbar und umgekehrt. \\
	Für $\lambda^n$-messbares $f: V\to [0,\infty]$ gilt: 
	\begin{align*}
		(\ast\ast\ast) && \int\limits_U f(\phi(x)) |\det{D\phi(x)} dx \geq \int\limits_V f(y) dy
	\end{align*}
	Denn zuerst folgt $(\ast\ast\ast)$ aus $\lambda\geq\mu$ mit $f = \psi_B$, indem wir $A = \phi^{-1}(B)$ wählen \\
	$\rightarrow (\ast\ast\ast)$ gilt für nichtnegative $\lambda^n$-Treppenfunktionen und damit für alle $f\geq 0$ mit Satz IV.9 und Satz IV.10.
	\item[Ziel:] Gleichheit in $(\ast\ast\ast)$ für $f\geq 0$ \\
	$\psi:V\to U$, $\psi = \phi^{-1}$ $C^1$-Diffeo \\
	$g = f\circ\phi |\det{D\phi}| $ \\
	Es gilt $1 = \det{D(\phi\circ\psi)}(y) = (\det{D\phi})(\psi(y)) \cdot \det{D\psi}(y)$ \\
	\begin{align*}
	\implies \int\limits_V f(y)dy &= \int\limits_V g(\psi(y)) |\det{D\psi(y)} dy \\
	&\geq \int\limits_U g(x) dx \\
	&= \int\limits_U f(\phi(x)) |\det{D\phi(x)} dx
	\end{align*}
	$\implies \int\limits_V f(y)dy = \int\limits_U f(\phi(x)) |\det{D\phi(x)}| dx \rightarrow$ (2) für $f\geq 0$ \\
	Gleichung (1) folgt darauf mit $f = \psi_{phi(A)}$  \\
	Für $f$ beliebig zerlege $f = f^+ -f^-$
  \end{proof}

  \sidenote{Vorlesung 20}{22.01.2021}
  \begin{example}
    \begin{enumerate}
      \item[]
      \item
        \begin{align*}
          f:\mathbb{R}^2 \to \mathbb{R}, f(x,y) = e^{-(x^2+y^2)} = e^{-||(x,y)||^2}\\
          \int\limits_{\mathbb{R}^2} f d\lambda^2 \stackrel{\text{Fubini}}{=} \int\limits_{\mathbb{R}} e^{-x^2}(\int\limits_{\mathbb{R}} e^{-y^2} dy) dx = (\int\limits_{\mathbb{R}} e^{-x^2} dx)^2
        \end{align*}
        Für Polarkoordinaten $\Phi:(0,\infty) \times (0,2\pi) \to \mathbb{R}^2 \setminus \{(x,0) \ | \ x \geq 0\}$ gilt: \\
        $$\det D\Phi(r,\Theta) = r$$\\
        Da $\{(x,0) \ | \ x \geq 0\}$ eine $\lambda^2$-Nullmenge ist, folgt aus der Transformationsformel:
        \begin{align*}
          \int\limits_{\mathbb{R}^2} f d\lambda^2
          &= \int\limits_{(0,\infty) \times (0,2\pi)} e^{-r^2} r \ d\lambda^2(r,\Theta)\\
          &\stackrel{\text{Fubini}}{=} \int\limits_0^{\infty} e^{-r^2}r(\int\limits_0^{2\pi}d\Theta)dr\\
          &= 2\pi \int\limits_0^{\infty} e^{-r^2} r \ dr\\
          &= 2\pi [-\frac{1}{2} e^{-r^2}]_{r=0}^{r=\infty}\\
          &= \pi\\
          \implies \int\limits_0^{\infty} e^{-x^2} dx &= \frac{\sqrt{\pi}}{2}
        \end{align*}
      \item Spezialfall $\Phi: \script{U} \to \script{V} C^1$-Diffeomorphismus ist Einschränkung einer linearen Abbildung.
        \begin{align*}
          &\implies \Phi(x) = Sx \text{ mit } S\in GL_n(\mathbb{R})\\
          &\implies D\Phi(x) = S \ \forall \ x \in \script{U}\\
          &\stackrel{Trafo}{\implies} \lambda^n(S(D)) = |\det S| \lambda^n(D) \text{ (siehe Satz ???)}\\
          \text{bzw. } \int\limits_{\script{V}} f(y) d\lambda^n(y) &= |\det S| \int\limits_{\script{U}} f(Sx) d\lambda^n(x)
        \end{align*}
      \item Polarkoordinaten im $\mathbb{R}^3$\\
        $$\Phi(r, \Theta, \phi) = (r \sin(\Theta) \cos(\phi), r \sin(\Theta)\sin(\phi), r \cos(\Theta))$$ ist $C^{\infty}$-Diffeomorphismus der offenen Mengen $\script{U} = (0,\infty) \times (0,\pi) \times (0,2\pi)$ und $\script{V} = \mathbb{R}^3 \setminus \{(x,0,z) \ | \ x \geq 0\}$\\
        Inverse:\\
        $r=\sqrt{x^2 + y^2 + z^2}, \Theta = \arccos(\frac{z}{r})$\\
        $\phi = \begin{cases}
          \arccos(\frac{x}{\sqrt{x^2 + y^2}}) & \text{, für } y \geq 0\\
          2\pi - \arccos(\frac{x}{\sqrt{x^2 + y^2}}) & \text{, für } y \leq 0
        \end{cases}$\\
        $D\Phi(r, \Theta, \phi) = \left(\begin{array}{ccc}
          \sin(\Theta)\cos(\phi) & r\cos(\Theta)\cos(\phi) & -r\sin(\Theta)\sin(\phi)\\
          \sin(\Theta)\sin(\phi) & r\cos(\Theta)\sin(\phi) & r\sin(\Theta)\cos(\phi)\\
          \cos(\Theta) & -r\sin(\Theta) & 0
        \end{array} \right)$
        \begin{align*}
          &\implies \det D\Phi = r^2 \sin(\Theta)\\
          E &:= [r_1, r_2] \times [\Theta_1, \Theta_2] \times [\phi_1, \phi_2]\\
          \lambda^3(\Phi(E)) &= \int\limits_{r1}^{r2}\int\limits_{\Theta_1}^{\Theta_2}\int\limits_{\phi_1}^{\phi_2} r^2 \sin(\Theta) d\phi d\Theta dr = \frac{r_2^3 - r_1^3}{3} (\cos(\Theta_1) - \cos(\Theta_2)) (\phi_2 - \phi_1)
        \end{align*}
    \end{enumerate}
  \end{example}

  \begin{remark}
    \underline{Ziel:} Umrechnung von Differentialoperatoren\\
    Begriff: $\Phi: \script{U} \to \script{V} \ C^k$-Diffeomorphismus zwischen $\script{U}, \script{V}$ offen.\\
    Gramsche Matrix $g \in C^{k-1}(\script{U}, \mathbb{R}^{n\times n}), g = (g_{i,j})$
    $$g(x) = D\Phi(x)^{\top} D\Phi(x) \text{ bzw.}\\
    g_{i,j}(x) = <\frac{\partial\Phi}{\partial x_i}(x), \frac{\partial\Phi}{\partial x_j}(x)>$$\\
    Für Polarkoordinaten im $\mathbb{R}^3$ gilt:
    $$(g_{i,j}(r, \Theta, \phi))_{1\leq i,j \leq 3} = \left(\begin{array}{ccc}
      1 & 0 & 0 \\
      0 & r^2 & 0 \\
      0 & 0 & r^2 \sin^2(\Theta)      
    \end{array}\right)$$
    Allgemein gilt:\\
    $g(x)$ ist symmetrisch und strikt positiv definit, denn
    $$<g(x)v,v> \ = \ <D\Phi(x)^{\top}D\Phi(x) v, v> \ = |D\Phi(x) v|^2 > 0$$
    für $v\neq 0$ und $D\Phi(x) \in GL_n(\mathbb{R}) \implies g(x)$ ist invertierbar.\\
    \\
    Wir setzen: $g^{ij}(x) = (g(x)^{-1})_{ij}$\\
    ... (Rest siehe Aufschrieb)
  \end{remark}

  \begin{theorem}
    Sei $\Phi \in C^1(\script{U}, \script{V})$ Diffeomorphismus zwischen $\script{U}, \script{V} \subseteq \mathbb{R}^n$ offen mit gramscher Matrix $(g_{ij})$
    \begin{enumerate}
      \item Für $v \in C^1(\script{V})$ gilt mit $\mu = v \circ \Phi$:
      $$(\triangledown v) \circ \Phi = D\Phi \cdot \triangledown_g u \ \ , \ \ \triangledown_g u := \sum\limits_{i,j=1}^n g^{ij} \frac{\partial u}{\partial x_i} e_j$$
      \item Für $y \in C^1(\script{V}, \mathbb{R}^n)$ gilt mit $y \circ \Phi = D\Phi x$:
        $$(div(y)) \circ \Phi = div_g x := \frac{1}{\sqrt{det(g)}} \sum\limits_{j=1}^n \frac{\partial}{\partial x_j} (\sqrt{det(g)} x_j)$$
      \item Ist $\Phi \in C^2(\script{U}, \script{V}), v \in C^2(\script{V}), u = v \circ \Phi$
        $$\implies (\triangle v) \circ \Phi = div_g \triangledown_g u = \frac{1}{\sqrt{det(g)}} \sum\limits_{i,j=1}^n \frac{\partial}{\partial x_i} (\sqrt{det(g)} g^{ij} \frac{\partial u}{\partial x_j})$$
    \end{enumerate}
  \end{theorem}
  \begin{proof}
    $< (\triangledown v) \circ \phi, \frac{\partial \phi}{\partial x_i} > = (Dv) \circ \phi \cdot (D\phi \cdot e_i) = D(v\circ \phi) \cdot e_i = Du \cdot e_i = \frac{\partial u}{\partial x_i} \\
    \implies$ 1) folgt aus $y\circ \phi = D\phi \cdot x$, $y = \triangledown v$ \\
    Allgemein gilt für ein Vektorfeld $X: \script{U} \to \mathbb{R}^n$ \\
    $<D\phi \cdot \triangledown_g u, D\phi\cdot x > = \sum\limits_{j=1}^n < D\phi \triangledown_gu, \frac{partial \phi}{\partial x_j} > x_j = \sum\limits_{j=1}^n \frac{\partial u}{\partial x_j} x_j$ \\
    \underline{2):}  $\psi \in C^1_c(V)$ bel. $\script{C} = \psi\circ\phi$ \\
    \begin{align*}
    	\int\limits_V \psi \div{y}dy &= -\int\limits_V <\triangledown \psi, y > dy \\
    	&= -\int\limits_{\script{U}} <(\triangledown \psi)\circ \phi, y\circ\phi > \sqrt{\det{g}} dx \\
    	&= -\int\limits_{\script{U}} < D\phi \cdot \triangledown_g\script{C}, D\phi\cdot x > \sqrt{\det{g}} dx \\
       &= -\int\limits_{\script{U}} \sum\limits_{j=1}^n \frac{\partial \script{C}}{\partial x_j} x_j \sqrt{\det{g}} dx \\
       &= \int\limits_{\script{U}} \script{C} \frac{1}{\det{g}} \frac{\partial}{\partial x_j} (\sqrt{\det{g}}x_j) \sqrt{\det{g}} dx \\
       &= \int\limits_{\script{U}} \script{C} div_g x \sqrt{\det{g}} dx \\
       &= \int\limits_V \psi div_g x\circ \phi^{-1} dy 
       \end{align*}
   	Übung Fundamentallemma Varationsrechnung \\
   	$\implies div{y\circ\phi} = div_g x \implies$ 2) \\
   	3) folgt aus Kombination von 1) und 2) mit $g^{ij} = g^{ji}$
  \end{proof}

  \begin{example}[Laplace in Polarkoordinaten im $\mathbb{R}^3$]
    \begin{align*}
      \triangledown_g u &= \frac{\partial u}{\partial r} e^r + \frac{1}{r^2} \frac{\partial u}{\partial \Theta} e^{\Theta} + \frac{1}{r^2 \sin^2(\Theta)} \frac{\partial u}{\partial \phi} e^{\phi}\\
      div_g x &= \frac{1}{r^2} \frac{\partial}{\partial r} (r^2 x^r) + \frac{1}{\sin(\Theta)} \frac{\partial}{\partial \Theta} (\sin(\Theta) x^{\Theta}) + \frac{\partial x^{\phi}}{\partial \phi}\\
      \triangle_g u &= \frac{1}{r^2} \frac{\partial}{\partial r} (r^2 \frac{\partial u}{\partial r})  + \frac{1}{r^2 \sin(\Theta)} \frac{\partial}{\partial \Theta} (\sin(\Theta) \frac{\partial u}{\partial \Theta}) + \frac{1}{r^2 \sin^2(\Theta)} \frac{\partial^2 u}{\partial \phi^2}
    \end{align*}
    $e^r, e^{\Theta}, e^{\phi}$ Standardbasis im $(r, \Theta, \phi)$-Raum und $x^r, x^{\Theta}, x^{\phi}$ sind zugehörige Koordinaten
    $$v(x,y,z) = (x^2 + y^2 + z^2)^{-\frac{1}{2}} \implies u = r^{-1}$$
    $$\implies \triangle v\circ \Phi = \triangle_g u = 0 \text{ auf }\mathbb{R}^3\setminus\{0\}$$
  \end{example}