\chapter{Das Flächenmaß auf Untermannigfaltigkeiten}
  \sidenote{Vorlesung 21}{25.01.2021}

  \begin{definition}
    Sei $\script{U} \subseteq \mathbb{R}^n$ offen. Eine Abbildung $f \in C^1(\script{U}, \mathbb{R}^{n+k})$ heißt \textbf{Immersion}, wenn gilt:
    $$Rang \ Df(x) = n \Leftrightarrow Ker \ Df(x) = \{0\} \ \forall \ x \in \script{U}$$
    $\frac{\partial f}{\partial x_1}(x), ..., \frac{\partial f}{\partial x_n}(x)$ bilden eine Basis von $Bild \ Df(x) \subseteq \mathbb{R}^{n+k}$. $n$ heißt \textbf{Dimension}, $k$ die \textbf{Kodimension} von $f$.\\
    Wir definieren die \textbf{Gramsche Matrix} oder \textbf{induzierte Metrik}
    $$g(x) = Df(x)^{\top} Df(x)$$ 
    $$\text{bzw. } g_{ij}(x) = <\frac{\partial f}{\partial x_i}|x|, \frac{\partial f}{\partial x_j}|x|>$$
    $\rightarrow (g_{ij})$ ist für $f \in C^1(\script{U}, \mathbb{R}^{n+k})$ beliebig definiert und positiv semidefinit.\\
    Die Matrix ist genau dann strikt positiv definit und damit invertierbar, wenn $f$ eine Immersion ist:
    $$<g(x)v, v> = |Df(x)v|^2 \geq 0$$
    $$\implies Ker \ g(x) = Ker \ Df(x)$$
  \end{definition}

  \begin{definition}{Flächenformel}
    $\script{U} \subseteq \mathbb{R}^n$ offen, $f \in C^1(\script{U}, \mathbb{R}^{n+k})$ eine $n$-dimensionale Immersion mit Gramscher Matrix $g$, und $E \subseteq \script{U}$ sei $\lambda^n$-messbar.\\
    Der ($n$-dimensionale) \textbf{Flächeninhalt} von $f$ auf $E$ ist definiert durch
    $$A(f,E) = \int\limits_E Jf(x) \ dx$$
    $$\text{mit } Jf = \sqrt{det \ g}$$
    $Jf$ heißt \textbf{Jacobische} von $f$.
  \end{definition}

  \newpage
  \begin{example}
    \begin{enumerate}
      \item[]
      \item $f = S \circ \Phi: \script{U} \to \mathbb{R}^{n+k}, \Phi \in C^1(\script{U}, \script{V})$ Diffeomorphismus zwischen $\script{U}, \script{V} \subseteq \mathbb{R}^n$ offen und $S: \mathbb{R}^n \to Y \subseteq \mathbb{R}^{n+k}$ lineare Isometrie, $E \subseteq \script{U}$ messbar, $Df(x) = S \ D\Phi(x)$
        \begin{align*}
          \implies A(f,E) 
          &= \int\limits_E \sqrt{det \ D\Phi(x)^{\top} S^{\top} S D\Phi(x)} dx\\
          &= \int\limits_E |det \ D\Phi(x)| dx \stackrel{Trafo}{=} \lambda^n(\Phi(E))
        \end{align*}
      \item 1-dim Immersion $f: I = (a,b) \to \mathbb{R}^n \ f=f(A)$ heißt \textbf{reguläre Kurve} Gramscher Matrix $g_{11}=<f'(A), f'(A)> = ||f'(A)||^2$\\
        Länge von Kurve $\rightarrow L(f) = \int\limits_a^b ||f'(A)|| dt$\\
        $$f(A) = (cos(t), sin(t)) \ \ \ L(f, [0, 3\pi]) = 3\pi$$
      \item 2-dim Immersion in $\mathbb{R}^n \ (n=2, k=1) \ f:\script{U}\to\mathbb{R}^3, f=f(x,y)$ heißt \textbf{reguläre Fläche}
        $$(g_{ij}) = \begin{pmatrix}
          ||\frac{\partial f}{\partial x}||^2 & <\frac{\partial f}{\partial x}, \frac{\partial f}{\partial y}>\\
          <\frac{\partial f}{\partial x}, \frac{\partial f}{\partial y}> & ||\frac{\partial f}{\partial y}||^2
        \end{pmatrix}$$ 
        $\implies Jf = \sqrt{||\frac{\partial f}{\partial x}||^2 ||\frac{\partial f}{\partial y}||^2 - <\frac{\partial f}{\partial x}, \frac{\partial f}{\partial y}>^2} = ||\frac{\partial f}{\partial x} \times \frac{\partial f}{\partial y}||$\\
        $(||a||^2||b||^2-<a,b>^2 = ||a \times b||^2 \text{ siehe LA})$\\
        \begin{align*}
          &\text{Polarkoordinaten:}\\
          &f: \script{U} = (0, \pi) \times (0, 2\pi) \to S^2\subseteq \mathbb{R}^3\\
          &f(\theta, \phi) = \begin{pmatrix}
            sin(\theta)cos(\phi) & sin(\theta)sin(\phi) & cos(\theta)
          \end{pmatrix}\\
          &\stackrel{\text{Kap. VII}}{\implies} Jf(\theta, \phi) = sin(\theta) \implies f \text{ reguläre Fläche}\\
          &A(f) = \int\limits_0^{\pi}\int\limits_0^{2\pi} sin(\theta) \ d\phi \ d\theta = 4\pi
        \end{align*}
      \item siehe Aufschrieb
    \end{enumerate}
  \end{example}

  \begin{theorem}
    $\Phi \in C^1(\script{U}, \script{V})$ Diffeomorphismus, $\script{U}, \script{V} \subseteq \mathbb{R}^n$ offen und $f \in C^1(\script{V}, \mathbb{R}^{n+k})$ Immersion. Dann gilt:
    $$A(f, \Phi(E)) = A(f\circ\Phi, E) \ \ \ \forall \ E \subseteq \script{U} \ \lambda^n\text{-messbar}$$
  \end{theorem}
  \begin{proof}
    Kettenregel \\
    $D(f\circ \phi)(x)^T D(f\circ\phi)(x) = D\phi(x)^T Df(\phi(x))^T Df(\phi(x)) D\phi(x)$ \\
    Determinanten davon ist \\
    $\det{(D(f\circ\phi)(x)^T D(f\circ\phi)(x))} = |\det{(D\phi(x))}|^2 \det{(Df(\phi(x))^T Df(\phi(x)))} \\
    \implies J(f\circ\phi)(x) = J f(\phi(x)) |\det{(D\phi(x))}|\\
    \implies $ Beh. folgt aus Trafo-Formel
  \end{proof}

  \begin{definition}[Untermannigfaltigkeiten]
    Eine Menge $M \subseteq \mathbb{R}^{n+k}$ heißt \textbf{Untermannigfaltigkeit} des $\mathbb{R}^{n+k}$, der Klasse $C^r,\\ r\in \mathbb{N}\cup\{\infty\}$ offen (mit $M \in \Omega$) und $\exists \ \Phi: \Omega \to \Phi(\Omega) \\
    C^r$-Diffeomorphismus mit
    $$\Phi(M \cap \Omega) = (\mathbb{R}^n \times \{0\}) \cap \Phi(\Omega)$$
    ($\Phi$ heißt \textbf{lokale Plättung} von $M$)\\
    \includegraphics[width=\textwidth]{img/VIII_4_UnterMGF.png}
  \end{definition}

  \begin{theorem}
    Sei $M \subseteq \mathbb{R}^{n+k}$. Dann sind äquivalent:
    \begin{enumerate}
      \item $M$ ist $n$-dimensionale Untermannigfaltigkeit der Klasse $C^r$
      \item $\forall p \in M \ \exists \ \Omega\subseteq\mathbb{R}^{n+k}$ offene Umgebung vvon $p$ und $f \in C^r(\Omega, \mathbb{R}^k)$ mit $M \cap \Omega = f^{-1}(0)$ und $Rang \ Df = k$ auf $\Omega$
      \item $\forall p \in M \ \exists$ offene Umgebungen $\script{U}\subseteq \mathbb{R}^n, V\subseteq \mathbb{R}^k$ und $g \in C^r(\script{U}, \script{V})$, so dass nach geeigneter Permutation der Koordinaten gilt:\\
      $$M \cap (\script{U} \times \script{V}) = \{(x, g(x)) \ | \ x \in \script{U}\}$$
      \item $\forall p \in M \ \exists$ offene Umgebung $\script{U}\subseteq \mathbb{R}^n$ und $\script{C} \in C^r(\script{U}, \mathbb{R}^{n+k})$ mit $\script{C}(x_0) = p$ für ein $x_0 \in \script{U}$ und $Rang \ D\script{C}(x) = n \ \forall \ x\in M$, so dass $\script{C}$ offene Teilmengen von $\script{U}$ in relativ offene Teilmengen von $M$ abbildet. 
    \end{enumerate}
  \end{theorem}

  \sidenote{Vorlesung 22}{29.01.2021}

  \begin{theorem}
    Jede $n$-dimensionale Untermannigfaltigkeit $M \subseteq \mathbb{R}^{n+k}$ ist als abzählbare Vereinigung $M = \bigcup\limits_{i\in\mathbb{N}} K_i$ von kompakten Mengen darstellbar.
  \end{theorem}

  \begin{proof}
    \item[I] $M\cap \overline{B_r(p)}$ ist kompakt für $p\in M$ und $r>0$ hinreichend klein. Sei dazu $\phi: W\to \phi(W)$ lokalte Plättung mit $p\in W$ \\
    $\phi: W\to \mathbb{R}^{n+k}$ ist stetig $\implies M\cap W = \phi^{-1}(\mathbb{R}^n\times\{0\})$ ist abgeschlossen in $W$. Für $\overline{B_r(p)}\subset W$ ist damit $M\cap \overline{B_r(p)}$ abgeschlossen, also kompakt. Setze nun für $p\in M$:  
    $$ r(p) = \sup\{r >0: M\cap \overline{B_r(p)} \text{ ist kompakt } \} \in (0,\infty] $$
    $\implies M\cap \overline{B_r(p)}$ ist kompakt $\forall r\in r(p)$ \\
    (Abg. Teilmengen einer kompakten Menge sind kompakt)
    \item[II] $r(p) \leq \liminf\limits_{i\to\infty}r(p_i)$ für $p,p_i\in M$ mit $p_i \to p$. \\
    Zu $r < r(p)$ wähle $R\in (r, r(p))$ \\
    Es gilt: $M\cap \overline{B_r(p)} \subset M\cap \overline{B_R(p)}$ für $i$ groß $\implies r(p_i) \geq r(p)$. \\
    Für $P\subset M$ dicht ist damit $M$ Vereinigung der kompakten Teilmengen \\ $M = \bigcup\limits_{p\in P} M\cap \overline{B_{\frac{r(p)}{2}}(p)}$ \\
    Betrachte nun $Q_{j,l} = 2^{-l} (j+[0,1]^{n+k})$ für $j\in \mathbb{Z}^{n+k}$, $l\in\mathbb{N}_0$. Wähle in jedem Würfel $Q_{j,l}$, der $M$ trifft, einen Punkt $p_{j,l}\in M$. Die Menge $P$ dieser $p_{j,l}$ ist abzählbar und dicht in $M$. 
  \end{proof}

  \begin{definition}
    Sei $M \subseteq \mathbb{R}^{n+k}$ eine $n$-dimensionale Untermannigfaltigkeit der Klasse $C^1$. Eine \textbf{lokale Parametrisierung} von $M$ ist eine injektive Immersion $f: \script{U} \to M \subseteq \mathbb{R}^{n+k}$, wobei $M \subseteq \mathbb{R}^n$ offen und $f \in C^1$.
  \end{definition}

  \begin{lemma}
    Für jede Untermannigfaltigkeit $M \subseteq \mathbb{R}^{n+k}$ der Klasse $C^1$ gibt es lokale\\
    $C^1$-Parametrisierungen $f_i:\script{U}_i \to M$, wobei $i\in\mathbb{N}$, sodass $M = \bigcup\limits_{i \in \mathbb{N}} f_i(\script{U}_i)$
  \end{lemma}
  \begin{proof}
    $\forall p\in M$ $\exists$ lokale Plättung $\phi: W \to \phi(W)$ mit $p\in W$ \\
    $\implies \script{U} = \mathbb{R}^n \cap \phi(W)$ ist offen in $\mathbb{R}^n$ und $f = \phi^{-1}|_{\mathbb{R}\cap \phi(W)}$ ist eine lokalte Parametrisierung von $M$ mit $p\in f(\script{U})$ \\
    Außerdem ist $f(\script{U}) = M\cap W$ offen in $M$. \\
    $K\subset M$ kompakt wird also durch endlich viele Bild(?) überdeckt. Beh. folgt aus Satz VIII.6
  \end{proof}

  \begin{theorem}
    $M \subseteq \mathbb{R}^{n+k}$ $C^1$-Untermannigfaltigkeit. Dann gelten:
    \begin{enumerate}
      \item Ist $f: \script{U} \to M$ lokale Parametrisierung von $M$, so ist $f(\script{U})$ offen in $M$ und $f: \script{U} \to f(\script{U})$ ist homeomorph, d.h. $f^{-1}$ ist stetig bzgl. euklidischer Metrik auf $f(\script{U})$.
      \item Sind $f_i:\script{U}_i \to f(\script{U}_i) = \script{V}_i$ für $i = 1,2$ lokale $C^1$-Parametrisierung von $M$, so ist $f_2^{-1}\circ f_1: f_1^{-1}(\script{V}_1\cap\script{V}_2) \to f_2^{-1}(\script{V}_1\cap\script{V}_2)$ ein $C^1$-Diffeomorphismus. 
    \end{enumerate}
  \end{theorem}
  \begin{proof}
    \item[1)] s. Blatt 11 (Satz VIII.5) \\
    \item[2)] O.E. $V_1 \cap V_2 \subset W$ für eine Plättung $\phi: W \to \phi(W)$ denn $C^1$-Eigenschaft ist lokal. \\
    Betrachte $\phi \circ f_i: f_i^{-1}(V_1\cap V_2) \to \phi(V_1\cap V_2)$ $i=1,2$ \\
    Es gilt: $f_2^{-1} \circ f_1 = (\phi\circ f_2)^{-1} \circ (\phi\circ f_1)$ auf $f_1^{-1}(V_1\cap V_2)$ \\
    \underline{z.z} $\phi\circ f_i$ ist Diffeo \\
    $\phi\circ f_i$ ist definiert, injektiv und es gilt $Rang (D(\phi\circ f_i)) (x) = Rang (D\phi(f_i(x)) Df(x)) = n$ $\forall x\in f_i^{-1}(V_1 \cap V_2) \\
    \overset{\text{Satz von der lokalten Umkehrbarkeit}}{\implies} \phi(V_1 \cap V_2)$ ist offen und $\phi\circ f_i: f_i^{-1}(V_1\cap V_2) \to \phi(V_1 \cap V_2)$ ist Diffeo. \\
    ($\phi(V_1\cap V_2)\subset \mathbb{R}^n\times \{0\} \subset \mathbb{R}^{n+k}$)
    
  \end{proof}

  \begin{theorem}[Flächenmaß]
    $M \subseteq \mathbb{R}^{n+k} C^1$-Untermannigfaltigkeit. Dann heißt $E \subseteq M$ messbar, falls gilt:
    $$f^{-1}(E) \text{ ist } \lambda^n\text{-messbar für jede lokale Parametrisierung } f: \script{U} \to M$$
    Das System $\script{M}$ der messbaren Teilmengen von $M$ ist eine $\sigma$-Algebra. Diese enthält die Borelmengen in $M$. Weiter gibt es genau ein Maß $\mu_M$ auf $\script{M}$, so dass für jede lokale Parametrisierung $f: \script{U} \to M$ und jedes $E \subseteq f(\script{U})$ messbar gilt:
    $$\mu_M(E) = \int\limits_{f^{-1}(E)}Jf(x) \ dx$$
  \end{theorem}
  \begin{proof}
    \item[Trivial:] $\varnothing \in M$ \\
    Für jede lok. Para. $f\script{U} \to M$ gilt: $f^{-1}(M\setminus E) = \script{U}\setminus f^{-1}(E)$ und $f^{-1}(\bigcup\limits_{i=1}^\infty E_i) = \bigcup\limits_{i=1}^\infty f^{-1}(E_i) \\
    \implies M$ ist $\sigma$-Algebra. \\
    Mit $V$ offen in $M$ ist $f^{-1}(V)$ offen im $\mathbb{R}^n \implies M$ enthält alle Borelmengen. \\
    Sei nun $M = \bigcup\limits_{i=1}^\infty M_i$ disjunkte Zerlegung in Mengen $M_i\in M$, sodass $M_i \subset V_i$ für lok. Para. $f_i: \script{U_i}\to f(\script{U_i}) = V_i$ \\
    Solch eine Zerlegung ex., denn nach lemma VIII.8 $\exists$ lok. para $f_i:\script{U_i} \to f(\script{U_i}) = V_i$ mit $M = \bigcup\limits_{i=1}^\infty V_i$ und wir wählen jetzt die Borelmengen $M_i = V_i \setminus \bigcup\limits_{j=1}^{i-1}V_j$. \\
    Aus den Eigenschaften folgt für das gesuchte Maß $\mu_M$ und alle $E\in M$ 
    \begin{align*}
    	(\ast) && \mu_M(E) = \sum\limits_{i=1}^\infty \mu_M(E\cap M_i) = \sum\limits_{i=1}^\infty \int\limits_{f_i^{-1}(E\cap M_i)} J f_i(x) dx 
    \end{align*}
	$\implies$ Eindeutigkeit von $\mu_M$. \\
	Durch $(\ast)$ wird ein Maß auf $M$ definiert. \\
	Ist $E = \bigcup\limits_{j=1}^\infty E_j$ mit $E_j$ messbar und paarweise disjunkt so folgt 
	\begin{align*}
		\mu_M(E) &= \sum\limits_{i=1}^\infty \int\limits_{f_i^{-1}(M_i\cap E)}Jf_i(x) dx  \\
		&= \sum\limits_{i,j=1}^\infty \int\limits_{f_i^{-1}(E_j\cap M_i)} Jf_i(x)dx \\
			&= \sum\limits_{j=1}^\infty \mu_M(E_j)
	\end{align*}
	Ist $f:\script{U} \to V = f(\script{U})$ beliebige lokale Para und $E\subset V$ messbar, so ist $\phi_i = f_i^{-1}\circ f: f^{-1}(V\cap V_i) \to f_i^{-1}(V\cap V_i)$ ein $C^1$-Diffeo zw. offenen Mengen nach Satz VIII.9(2). Aus Satz VIII.3 folgt \begin{align*}
		\mu_M(E) &= \sum\limits_{i=1}^\infty \int\limits_{f_i^{-1}(E\cap M_i)} Jf_i(x) dx \\
		&= \sum\limits_{i=1}^\infty \int\limits_{\phi_i\circ f^{-1}(E\cap M_i)} Jf_i(x) dx \\
		&= \sum\limits_{i=1}^\infty \int\limits_{f^{-1}(E\cap M_i)}J(f_i\circ\phi_i)(x) dx \\
		&= \int\limits_{f^{-1}(E)} Jf(x) dx
	\end{align*}

  \end{proof}
	\newpage
  \begin{theorem}[Oberflächenintegral]
    Sei $M \subseteq \mathbb{R}^{n+k} n$-dimensionale $C^1$-Untermannigfaltigkeit und $M = \bigcup\limits_{i \in \mathbb{N}} M_i$ eine paarweiße disjunkte, messbare Zerlegung mit $M_i \subseteq \script{V}_i$ für lokale Parametrisierungen $f_i: \script{U}_i \to \script{V}_i$. Für eine messbare Funktion $u: M \to \bar{\mathbb{R}}$ gilt:
    $$\int\limits_M u \ d\mu_M = \sum\limits_{i \in \mathbb{N}} \int\limits_{f_i^{-1}(M_i)} u(f_i(x))\ Jf_i(x)\ dx$$
  \end{theorem}
  \begin{proof}
  	Aussage gilt nach Satz VIII.10 für $u$ = Charakteristische Funktion\\  $\implies$ gilt auch für messbare Treppenfunktionen. \\
  	Für $u\geq 0$ folgt Beh. durch Approximation von unten durch Treppenfunktionen (Satz Mon. Konv.) [Auf der rechten Seite benutze zuerst Mon. Konv. für die einzelnen Integrale und dann für die Reihe] \\ Für $u$ integrierbar zerlege $u = u^+ - u^-$
  \end{proof}

  \sidenote{Vorlesung 23}{01.02.2021}
  \begin{lemma}
    Sei $T:\mathbb{R}^{n+k} \to \mathbb{R}^{n+k}$ eine \textbf{Ähnlichkeitsabbildung}, d.h. $\exists \lambda>0, Q\in O(n+k)$ und $a\in\mathbb{R}^{n+k}$ mit $T(p) = \lambda Q(p+a)$. Ist $M \subseteq \mathbb{R}^{n+k}$ eine $n$-dimensionale $C^1$-Untermannigfaltigkeit, so auch $N = T(M)$ und für $\mu_M$ bzw. $\mu_N$ gilt:
    \begin{enumerate}
      \item Ist $A \subseteq M$ messbar $\implies T(A) \subseteq N$ messbar und 
        $$\mu_N(T(A)) = \lambda^n (\mu_M(A))$$
      \item Ist $u: N \to \bar{\mathbb{R}} \ \mu_N$-messbar, so ist $u \circ T: M \to \bar{\mathbb{R}} \ \mu_M$-messbar und es gilt, sofern eines der Integrale existiert:
        $$\int\limits_N u(q) \ d\mu_N(q) = \lambda^n \int\limits_M u(T(p)) \ d\mu_M(p)$$
    \end{enumerate}
  \end{lemma}
  \begin{proof}
    Zu $q\in N$ wähle lokalte Plättung $\phi: W \to \phi(W)$ von $M$ mit $T^{-1}(q) \in W$. Dann ist $\phi \circ T^{-1}: T(W) \to \phi(W)$ eine lokalte Plättung von $N$ $\implies N$ ist $C^1$-Umge. \\
    Ist $f: \script{U} \to M$ lokale Parametrisierung, so auch $T\circ f: \script{U}\to W$ und umgekehrt. Wegen $(T\circ f)^{-1} (T(A)) = f^{-1}(A)$ ist $A\subset M$ $\mu_M$-messbar $\Leftrightarrow T(A) \subset N$ $\mu_M$-messbar. 
    \item[1)] O.E. $A\subset f(\script{U})$ (s. Beweis zu Satz VIII.10) für $f: \script{U}\to M$ lokalte Parametriersung. $DT(x) = \lambda Q$ \\
    $\implies D(T\circ f)(x)^T D(T\circ f)(x) = Df(x)^T DT(f(x))^T DT(f(x))Df(x) = \lambda^n Df(x)^T Df(x)$ \\
    $\overset{\text{Satz VIII.11}}{\implies} \mu_M (T(A)) = \int\limits_{(T\circ f)^{-1}(T(A))} J(T\circ f)(x) dx = \lambda^n\int\limits_{f^{-1}(A)}Jf(x) dx = \lambda^n \mu_M(A)$ 
    \item[2)] Für $u = \psi_B$ mit $B\subset N$ $\mu_N$-messbar folgt 2) aus 1).
    Durch Approximation mit Treppenfunktionen von unten folgt 2) für $u\geq 0$ und für $u$ belibig zerlege $u = u^+ - u^-$
  \end{proof}

  \begin{theorem}[Zwiebelformel]
    Für $u \in L^1(\mathbb{R}^{n+1})$ ist $u|_{\partial B_r(0)} \subseteq L^1(\mu_{\partial B_r(0)})$ für fast alle $r>0$ und es gilt:
    \begin{align*}
      \int\limits_{\mathbb{R}^{n+1}} u(p) \ dp
      &= \int\limits_0^{\infty} \int\limits_{\partial B_r(0)} u(p) \ d\mu_{\partial B_r(0)}(p)\ dr\\
      &= \int\limits_0^{\infty} r^n \int\limits_{S^n} u(rw) \ d\mu_{S^n}(w)\ dr
    \end{align*}
  \end{theorem}
  \begin{proof}
    Sei $f: \script{U}\to V\subset S^n$ lok. Para. und $C(V) = \{rw: w\in V, r> 0 \}$ sei der offene Kegel über $V$. \\
    Betrachte Diffeo $\phi: (0,\infty) \times \script{U} \to C(V)$, $\phi(v,x) = rf(x)$. \\
    Mit $g_{i,j}(x) = < \frac{\partial f(x)}{\partial x_i}, \frac{\partial f(x)}{\partial x_j}>$ gilt $D\phi (r,x)^T D\phi (r,x) = \begin{pmatrix} 1 & 0 \\ 0 & r^2 g(x) \end{pmatrix} \in \mathbb{R}^{n+1}\times \mathbb{R}^{n+1}$ \\
    Ist $E = f(A)$ für $\lambda^n$-messbares $A\subset \script{U}$ und $C(E)$ der Kegel über $E$, so folgt 
    \begin{align*}
    	\int\limits_{C(E)} u(p) d\lambda^{n+1}(p) &= \int\limits_{(0,\infty)\times A} u(rf(x)) r^n \sqrt{det{(g(x))}} d\lambda^{n+1}(r,x) \\
    	&= \int\limits_0^\infty r^n \int\limits_A u(rf(x)) \sqrt{\det{(g(x))}} d\lambda^n(x) d\lambda^1(r) \\
    	&= \int\limits_0^\infty r^n \int\limits_E u(rw) d\mu_{S^n}(w) d\lambda^1(r) \\
    	&= \int\limits_0^\infty \int\limits_{\{rw: w\in E\} } u(p) d\mu_{\partial B_r(0)}(p) d\lambda^1(r)
    \end{align*}
	Wähle nun disjunkte Zerlegung $S^n = \bigcup\limits_{j=1}^N E_j$ mit $E_j \subset V_j$, wobei $f_j: \script{U_j}\to V_j$ lok. Para. ist. Durch Addition folgt Beh. 
  \end{proof}

  \begin{example}
    Mit $u = \psi_{B_1(0)}$ folgt für $w_n = \mu_{S^n}(S^n)$:
    $$\alpha_{n+1} = \lambda^{n+1}(B_1(0)) = \int\limits_0^1 \mu_{\partial B_r(0)} (\partial B_r(0)) \ dr = \int\limits_0^1 w_n r^n \ dr = \frac{w_n}{n+1}$$
    $\implies w_n = (n+1) \alpha_{n+1}$\\
    z.B. $w_1 = 2\pi, w_2 = 4\pi, w_3 = 2\pi^2, ...$
  \end{example}